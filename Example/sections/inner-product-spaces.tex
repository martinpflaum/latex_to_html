% Copyright 2017 Markus J. Pflaum, licensed under CC BY-NC-ND 4.0
% main author: 
%   Markus J. Pflaum
%
\section{Inner product spaces}
\label{sec:inner-product-spaces}
%
%
\para 
Let us first remind the reader that as before $\fldK$ stands for the field 
of real or of complex numbers. We will keep this notational agreement 
throughout the whole chapter.
\begin{definition}
By a \emph{sesquilinear form} on a $\fldK$-vector space $\vectorspV$ one understands a map
$\langle \cdot , \cdot \rangle : \vectorspV \times \vectorspV \to \fldK$
with the following two properties:
\begin{enumerate}[label={\textup{({\sffamily SF\arabic*})}},leftmargin=*]
\item
\label{axiom:form-linearity-in-first-coordinate} 
  The map $\langle \cdot , \cdot \rangle$ is \emph{linear} in its first coordinate which means that
  \[
    \langle v_1 + v_2, w \rangle = \langle v_1, w \rangle + \langle v_2, w \rangle
    \quad \text{and} \quad \langle r v, w \rangle = r \langle v, w \rangle
  \]
  for all  $v, w , v_1,v_2 \in \vectorspV$ and $r\in \fldK$.
\item
 \label{axiom:form-conjugate-linearity-in-second-coordinate} 
  The map $\langle \cdot , \cdot \rangle$ is \emph{conjugate-linear} in its second coordinate which means that
  \[
    \langle v_1 + v_2, w \rangle = \langle v_1, w \rangle + \langle v_2, w \rangle
    \quad \text{and} \quad \langle  v, r w \rangle = \overline{r} \langle v, w \rangle
  \]
  for all  $v, w , v_1,v_2 \in \vectorspV$ and $r\in \fldK$.
\end{enumerate}
A \emph{hermitian form} is a sesquilinear form $\langle \cdot , \cdot \rangle$ on $\vectorspV$ with the following
additional property:
\begin{enumerate}[label={\textup{({\sffamily SF\arabic*})}},leftmargin=*,resume]
\item
\label{axiom:form-conjugate-symmetry} 
  The map $\langle \cdot , \cdot \rangle$ is \emph{conjugate symmetric} which means that 
  \[
    \langle v, w \rangle = \overline{\langle w, v \rangle} \quad \text{for all  } v, w \in \vectorspV \ .
  \]
\end{enumerate}
A sesquilinear form  $\langle \cdot , \cdot \rangle$ is  called
\emph{weakly-nondegenerate} if it satisfies axiom
\begin{enumerate}[label={\textup{({\sffamily SF\arabic*w)}}},leftmargin=*,resume]
\item
\label{axiom:form-weakly-nondegenerate} 
  For every $v\in \vectorspV$, the map $\vectorspV \to \fldK$, $w \to \langle w,v \rangle$ is the zero 
  map if and only if $v=0$.
\end{enumerate}
Finally, one calls a hermitian form  $\langle \cdot , \cdot \rangle$ on $\vectorspV$   
\emph{positive semidefinite} if
\begin{enumerate}[label={\textup{({\sffamily SF\arabic*s})}},leftmargin=*,resume]
\item
\label{axiom:form-positive-semidefinite} 
  $\langle v,v \rangle \geq 0$ for all $v\in \vectorspV$.
\end{enumerate}
\end{definition}

\begin{remark}
Recall that a map  $\langle \cdot , \cdot \rangle : \vectorspV \times \vectorspV \to \fldK$ is called \emph{bilinear},
if it satisfies \ref{axiom:form-linearity-in-first-coordinate} and 
\begin{axiomlist}[BF]
\item
 \label{axiom:form-linearity-in-second-coordinate} 
  The map $\langle \cdot , \cdot \rangle$ is \emph{linear} in its second coordinate which means that
  \[
    \langle v_1 + v_2, w \rangle =  \langle  v_1, w \rangle + \langle v_2, w \rangle
    \quad \text{and} \quad \langle  v, r w \rangle = r \langle v, w \rangle
  \]
  for all  $v, v_1,v_2, w \in \vectorspV$ and $r\in \fldK$.
\end{axiomlist}
If the underlying ground field $\fldK$ coincides with the field of real numbers, 
a sesquilinear form is by definition the same as a bilinear form,
and a hermitian form the same as a symmetric bilinear form. 
\end{remark}

\para \label{para:properties-seminorm-associated-positive-semidefinite-hermitian-form}
Given a positive semidefinite hermitian form $\langle \cdot , \cdot \rangle$ on a 
$\fldK$-vector space $\vectorspV$, one calls two vectors $v,w \in \vectorspV$ \emph{orthogonal} 
if $\langle v , w \rangle = 0$. Since the hermitian form 
$\langle \cdot , \cdot \rangle$ is assumed to be positive semidefinite, the map 
\[
 \| \cdot \| : \vectorspV \to \R_{\geq 0} , \: v \mapsto \|v \| = \sqrt{\langle v , v \rangle} 
\]
is well-defined. We will later see that $\| \cdot \|$  is a seminorm on $\vectorspV$
and therefore call the map $\| \cdot \|$ the \emph{seminorm associated to}
$\langle \cdot , \cdot \rangle$. The following formulas are immediate consequences of the 
properties defining a positive semidefinite hermitian form and the definition
of the associated seminorm: 
\begin{align} 
  \label{eq:norm-squared-sum}
  & \| v+w \|^2  = \| v\|^2 + 2\, \Re \, \langle  v ,  w \rangle + \| w\|^2 \quad \text{for all } 
  v,w \in \vectorspV \ , \\
  \label{eq:pythagorean-theorem}
  & \| v+w \|^2  = \| v\|^2 + \| w\|^2 \quad \text{for all orthogonal } v,w \in \vectorspV \ ,  \\
  \label{eq:parallelogram-identity} 
  & \| v+w \|^2  + \|v-w\|^2  = 2 \big( \| v\|^2 + \| w\|^2 \big) \quad \text{for all } v,w \in \vectorspV\ ,\\
  \label{eq:absolute-homogeneity} 
  & \| rv \| = \sqrt{ |r|^2 \langle  v ,  v \rangle }  = |r| \| v \| 
  \quad \text{for all } v,w \in \vectorspV \text{ and } r\in \fldK \ .
\end{align} 
Formula \eqref{eq:pythagorean-theorem} is an abstract version of the \emph{pythagorean theorem},
\Cref{eq:parallelogram-identity} is called the \emph{parallelogram identity}. 
The triangle inequality for the map $\| \cdot \|$  will turn out to be a consequence of the 
following result. 

\begin{proposition}[Cauchy--Schwarz inequality]
\label{thm:cauchy-schwartz-inequality-inner-product-spaces}
%
Given a positive semidefinite hermitian form $\langle \cdot , \cdot \rangle$ on a $\fldK$-vector space 
$\vectorspV$ the following inequality holds true:
\begin{equation}
\label{eq:cauchy-schwarz-inequality}
  |\langle v, w \rangle| \leq \|v\|\|w\| \quad \text{for all $v,w \in \vectorspV$}.
\end{equation}
Equality holds if and only if $v$ and $w$ are linearly dependant.
\end{proposition}
\begin{proof}
If $v$ or $w$ is $0$, the proof is trivial, hence we can assume both to be nonzero. Put
\[
  c = - \frac{\langle v, w \rangle}{\|v\|^2}
\]
and compute
\begin{equation}
\label{eq:expanding-norm-of-sum-in-inner-product-space}
\begin{split}
   0 & \leq \|c v + w\|^2 = \langle c v + w, c v + w \rangle = 
   |c|^2 \langle v, v \rangle + c \langle v, w \rangle + 
   \overline{c}\langle w, v \rangle + \langle w, w \rangle = \\
   & =  |c|^2\|v\|^2 + 2 \, \Re \big( c\langle v, w \rangle \big) + \|w\|^2 = 
   \frac{|\langle v, w \rangle|^2}{\|v\|^2} - 2\frac{|\langle v, w \rangle|^2}{\|v\|^2} + \|w\|^2 = \\
   & = \|w\|^2 - \frac{|\langle v, w \rangle|^2}{\|v\|^2} \ .
 \end{split}
\end{equation}
Hence
\[
   |\langle v, w \rangle|^2 \leq \|v\|^2\|w\|^2 
\]
which entails the Cauchy--Schwarz inequality. 

If $v,w$ are linearly dependant nonzero elements of $\vectorspV$, then there exists a nonzero scalar 
$a\in \fldK$ such that $v = a w$. Hence 
\[
  |\langle v , w \rangle| = |a| \, \| w \|^2 
  = \| v \|  \| w \| \, .
\]
Vice versa, if equality holds, then \Cref{eq:expanding-norm-of-sum-in-inner-product-space} 
entails that $c v +w  = 0$, which means that $v$ and $w$ are linearly dependant. 
\end{proof}

\begin{lemma}
\label{thm:positive-definitess-equivalent-nondegeneracy}
  A positive semidefinite hermitian form $\langle \cdot , \cdot \rangle$ on a $\fldK$-vector space 
  $\vectorspV$ is weakly-nondegenerate if and only if it is \emph{positive definite} that is if and only if
  \begin{enumerate}[label=\textup{({\sffamily SF\arabic*p)}},leftmargin=*]
  \setcounter{enumi}{4}
  \item
  \label{axiom:form-positive-definite} 
  $\langle v,v \rangle > 0$ for all $v\in \vectorspV \setminus \{ 0 \}$.
  \end{enumerate}
\end{lemma}

\begin{proof}
  A positive definite real bilinear or complex  hermitian form $\langle \cdot , \cdot \rangle$ is 
  non-degenerate since for every $v \in \vectorspV \setminus \{ 0 \}$ the  linear form
  $\langle - ,v \rangle : \vectorspV \to \fldK$ then is  nonzero  because  $\langle v,v \rangle > 0$.

  Conversely, if $\langle - ,v \rangle : \vectorspV \to \fldK$ is nonzero for all 
  $v \in \vectorspV \setminus \{ 0 \}$, then  there exists an element $w \in \vectorspV$ such that 
  $\langle w,v \rangle \neq 0$. The Cauchy--Schwarz inequality entails
  \[
    0 < |\langle w,v \rangle |^2 \leq \langle w,w \rangle \, \langle v,v \rangle \ ,
  \]
  which implies $\langle v,v \rangle > 0$. Hence $\langle \cdot , \cdot \rangle$ is positive definite.
\end{proof}


\begin{proposition}
 The map 
 \[
  \| \cdot \| :V \to \R_{\geq 0} , \: v \mapsto \|v \| = \sqrt{\langle v , v \rangle} 
 \]
 associated to a positive semidefinite hermitian form 
 $\langle \cdot , \cdot \rangle$ on a $\fldK$-vector space $\vectorspV$ is a seminorm. 
 If the hermitian form is positive definite, then $\| \cdot \|$ is even  a norm. 
\end{proposition}
\begin{proof}
 Absolute homogeneity \ref{axiom:norm-absolute-homogeneity} is given by Eq.~\eqref{eq:absolute-homogeneity}.
 The triangle inequality is a consequence of the Cauchy--Schwarz inequality:
 \[
   \| v + w \|^2 =   \| v\|^2 + 2 \, \Re \, \langle  v ,  w \rangle + \| w\|^2 \leq
    \| v\|^2 + 2 \, \|  v \| \, \| w \| + \| w\|^2 = \big(\| v\|+\| w\| \big)^2 \ .
 \] 
 Finally, if $\langle \cdot , \cdot \rangle$ is positive definite, then 
 $\| v \| = \sqrt{\langle v , v \rangle} > 0 $ for all $v \in \vectorspV \setminus \{ 0 \}$,
 so $\| \cdot \|$ is  a norm.
\end{proof}


%%%%%%%%%%%%%%%%%%%
%% Hilbert space definition
%%%%%%%%%%%%%%%%%%%
\begin{definition} 
\label{def:hilbert-space}
By an \emph{inner product} or a \emph{scalar product} on a $\fldK$-vector space $\hilbertH$ one 
understands a positive definite hermitian form on $\hilbertH$. A $\fldK$-vector space $\hilbertH$ 
together with an inner product $\langle \cdot , \cdot \rangle : \hilbertH \times \hilbertH \to \fldK$ is 
called an \emph{inner product space} or a \emph{pre-Hilbert space}. 

A hermitian form on a  $\fldK$-vector space $\hilbertH$ which is only positive semidefinite is called a \emph{semi-inner product} or a \emph{semi-scalar product}.

A \emph{Hilbert space} is an inner product space  $(\hilbertH , \langle \cdot , \cdot \rangle)$  which is 
complete as a normed vector space. In other words, a Hilbert space is Banach space where the 
norm on the space is induced by an inner product.
\end{definition}

\begin{examples}
\label{ex:inner-product-spaces}
\begin{environmentlist}
\item The vector space $\R^n$ with the \emph{euclidean inner product} 
      \[ 
        \langle \cdot , \cdot \rangle : \R^n \times \R^n \to \R, \:
        \big( (v_1,\ldots , v_n),(w_1,\ldots , w_n) \big) \mapsto 
        \sum_{i=1}^n v_i w_i  
      \]
      is a real Hilbert space. Obviously, $\langle \cdot , \cdot \rangle$ is linear in the first argument,
      symmetric, and positive definite, hence a real inner product. The associated norm is 
      the \emph{euclidean norm}. We have seen before that $\R^n$ with the euclidean norm is complete.      
\item The vector space $\C^n$ together with the hermitian form 
      \[
        \langle \cdot , \cdot \rangle : \C^n \times \C^n \to \C, \:
        \big( (v_1,\ldots , v_n),(w_1,\ldots , w_n) \big) \mapsto 
        \sum_{i=1}^n v_i\overline{w}_i
      \]
      is a complex Hilbert space. One immediately verifies that $\langle \cdot , \cdot \rangle$ is linear in the first argument,
      conjugate symmetric, and positive definite. Hence $\langle \cdot , \cdot \rangle$ is a complex inner product which we 
      sometimes call the \emph{standard hermitian inner product} on $\C^n$.  Its associated norm is again the euclidean 
      norm, so by completeness of $\C^n\cong \R^{2n}$ endowed with the euclidean norm one obtains the claim.      
\item The set 
      \[
        \ell^2 = 
        \left\{(z_k)_{k\in \N} \in \C^\N \suchthat \sum_{k=0}^\infty |z_k|^2 < \infty \right\}
      \]
      of square summable sequences of complex numbers  is a complex Hilbert space 
      with inner product
      \[
        \langle \cdot , \cdot \rangle :
        \ell^2 \times \ell^2 \to \C, \: \big((z_k)_{k\in \N},(w_k)_{k\in \N} \big)
        \mapsto \sum_{k=0}^\infty z_k \overline{w}_k \ .
      \]
      To prove this one needs to first verify that $\ell^2$ is a subvector space of $\C^\N$. 
      For $z = (z_k)_{k\in \N} \in \C^\N$ denote by $\| z\|$ the \emph{extended norm} 
      $\sqrt{\sum_{k=0}^\infty |z_k|^2} = \sup\limits_{K\in \N} \sqrt{\sum_{k=0}^K |z_k|^2} \in \closedint{0,\infty}$.
      Then $z\in \ell^2$ if and only if $\|z\| < \infty$. Now let $a \in \C$ and $z\in \ell^2$ and compute
      \[
         \| a z \| = \sqrt{\sum_{k=0}^\infty |a  z_k|^2} = |a| \, \sqrt{\sum_{k=0}^\infty |z_k|^2} = |a| \cdot \| z \| < \infty \ .
      \] 
      Hence $az \in \ell^2$. If $z,w \in \ell^2$, denote for each $K\in \N$ by $z_{(K)}$ and $w_{(K)}$ the ``cut-off'' 
      vectors $(z_0, \ldots , z_K) \in \C^{K+1}$ and  $(w_0, \ldots , w_K) \in \C^{K+1}$, respectively. 
      By the triangle inequality for the norm on the Hilbert space $\C^{K+1}$  one concludes 
      \[
         \sqrt{\sum_{k=0}^K  | z_k + w_k |^2} =  \| z_{(K)} + w_{(K)} \| \leq 
         \| z_{(K)} \| + \| w_{(K)} \| \leq \| z \| + \| w \| < \infty \ .
      \] 
      Therefore, the sequence of partial sums $\sum_{k=0}^K  | z_k + w_k |^2$, $K\in \N$, is bounded,
      so convergent by the the monotone convergence theorem. One  obtains 
       \[
          \| z + w \| = \lim_{K\to\infty} \sqrt{ \sum_{k=0}^K  | z_k + w_k |^2} \leq \| z \| + \| w \| < \infty \ .
      \]
      Hence  $z + w$ is square summable and $\ell^2$ a vector subspace of $\C^\N$ indeed. 
      Note that our argument also shows that the restriction of the extended norm to $\ell^2$ is a norm. 

      We need to show that $\langle \cdot , \cdot \rangle$ is well-defined. 
      To this end it suffices to prove that for all $z,w \in\ell^2$ the family $\left( z_k \overline{w}_k\right)_{k\in \N}$ 
      is absolutely summable or in other words that $\sum_{k=0}^\infty \left| z_k \overline{w}_k \right| < \infty$. 
      One concludes by the H\"older inequality for  sums  
      \[
        \sum_{k=0}^K \left| z_k  \overline{w}_k \right| = \sum_{k=0}^K \left| z_k  w_k \right| 
        % =  \left| \inprod{(|z_0|,\ldots,|z_N|),(|w_0|,\ldots,|w_N|)  } \right|
        \leq \| z_{(K)} \| \,  \| w_{(K)} \| \leq  \| z \| \,  \| w \| \ .
      \]
      So the left hand side has an upper bound uniform in $K$ which by the monotone convergence theorem entails
      convergence of the partial sums and the estimate 
      \[
         \sum_{k=0}^\infty \left| z_k \overline{w}_k \right| \leq  \| z \| \,  \| w \| < \infty \ . 
      \] 
      By definition it is clear that $ \langle \cdot , \cdot \rangle $ is linear in the first argument, 
      conjugate symmetric and positive definite, hence a complex inner product. Note that the norm 
      associated to  $ \langle \cdot , \cdot \rangle $ coincides with the above defined map $\| \cdot \|$. 

      It remains to be shown that $\ell^2$ is complete. Let $(z^n)_{n\in \N}$ with $z^n = {(z^n_k)}_{k\in \N}\in \ell^2$
      for all $n\in \N$ be a Cauchy sequence in $\ell^2$.    
      For $\varepsilon >0$ choose $N_\varepsilon \in \N$ so that
      \[
        \| z^n - z^m   \|< \varepsilon \quad \text{for all } n,m \geq N_\varepsilon \ . 
      \]
      For each fixed $k\in \N$ one therefore has 
      \begin{equation}
        \label{eq:estimate-component-sequence-Cauchy-sequence-square-summable-sequences}
        | z^n_k - z^m_k | \leq \| z^n - z^m   \|< \varepsilon \quad \text{for all } n,m \geq N_\varepsilon \ . 
      \end{equation}
      By completeness of $\C$ there exist $z_k \in \C$ such that $\lim_{n\to\infty} z^n_k = z_k$ for all $k\in \N$. 
      We claim that $z = (z_k)_{k\in \N}$ is an element of $\ell^2$ and that $(z^n)_{n\in \N}$ converges to $z$.
      To verify this observe that for all $\varepsilon >0$, $K\in \N$ and $n\geq N_\varepsilon$
      \[
        \sum_{k=0}^K  | z_k - z^n_k |^2 = \lim_{m\to\infty}  \sum_{k=0}^K  | z^m_k - z^n_k |^2 
        \leq \sup_{m \geq N_\varepsilon} \sum_{k=0}^K  | z^m_k - z^n_k |^2 
        \leq \sup_{m \geq N_\varepsilon} \| z^m - z^n \|^2 \leq \varepsilon^2 \ . 
      \]
      This implies by the triangle inequality and the fact that the Cauchy sequence $(z^n)_{n\in \N}$ is bounded in norm by some 
      $C>0$ that for all $K\in \N$ and $N = N_1$
      \[
        \sqrt{\sum_{k=0}^K  | z_k|^2 } = \| z_{(K)} \| \leq  \| z_{(K)} - z^N_{(K)} \| + \| z^N_{(K)} \| 
        \leq  \| z_{(K)} - z^N_{(K)} \| + \| z^N \| 
        \leq 1 + C \ . 
      \]
      Hence $ \| z \|= \sqrt{\sum_{k=0}^\infty | z_k|^2 } \leq 1 + C $  and $z \in \ell^2$. In addition one obtains
      \[
       \| z - z^n \| = \lim_{K\to\infty} \sqrt{\sum_{k=0}^K  | z_k - z^n_k |^2} \leq \varepsilon \quad \text{for all } n \geq N_\varepsilon \ . 
      \]
      This means that $z$ is the limit of the sequence $(z^n)_{n\in \N}$ and $\ell^2$ is complete. 
\item Let 
      \[ 
        \mathscr{L}^2(\R^n) = \left\{ f : \R^n \to \C \suchthat f
        \text{ is Lebesgue measurable and }
        \int_{\R^n}|f|^2 d\lambda < \infty \right\}
      \] 
      denote the space of Lebesgue square integrable functions on $\R^n$. 
      Then the map    
      \[
          \langle \cdot , \cdot \rangle : \mathscr{L}^2(\R^n) \times  \mathscr{L}^2(\R^n) \to \C, \: 
         (f,g) \mapsto \int_{\R^n}f\overline{g}\, d\lambda
      \]
      is a positive semidefinite hermitian form on $\mathscr{L}^2(\R^n)$. 
      Modding out $\mathscr{L}^2(\R^n)$ by the kernel 
      \[ 
        \mathscr{N} := \Ker (\| \cdot \|) = 
        \left\{ f \in \mathscr{L}^2(\R^n) \suchthat \int_{\R^n}|f|^2 d\lambda = 0 \right\}
      \]
      gives the Lebesgue space
      \[
          L^2 (\R^n) := \mathscr{L}^2(\R^n) / \mathscr{N} \ .
      \]
      
      The hermitian form $\langle \cdot , \cdot  \rangle $ vanishes on $\mathscr{N} \times  \mathscr{L}^2(\R^n)$ and 
      $\mathscr{L}^2(\R^n) \times \mathscr{N}$ by the Cauchy--Schwarz inequality, hence descends to a hermitian form
      \[
        \langle \cdot , \cdot \rangle : L^2(\R^n) \times  L^2(\R^n) \to \C, \:
        (f + \mathscr{N} ,g+ \mathscr{N}) \mapsto \int_{\R^n}f\overline{g}\, d\lambda \ .
      \]
      That hermitian form is positive definite, since 
      $\langle f + \mathscr{N}, f + \mathscr{N} \rangle = 0$ means 
      $ \int_{\R^n}|f|^2 d\lambda =0$, hence $f\in \mathscr{N}$.
      So $L^2(\R^n)$ together with $\langle \cdot , \cdot \rangle $ is a 
      Hilbert space which we call the 
      \emph{Hilbert space of square-integrable functions} on $\R^n$. 
      \add{provide details on well-definedness of hermitian form and show completeness}     
\end{environmentlist}
\end{examples}

%%%%%%%%%%%%%%%%%%%
%% Theorem giving condition to relate inner product with norm
%%%%%%%%%%%%%%%%%%%

\begin{theorem} 
\label{thm:parallegram-identity-guarantees-that-norm-comes-from-some-inner-product}
Let $\vectorspV$ be a normed $\fldK$-vector space. Then the norm $\|\cdot\| : \vectorspV \to \R_{\geq 0}$
is associated to an inner product $\langle \cdot , \cdot \rangle :  \vectorspV \times \vectorspV \to \fldK$ 
if and only if the \emph{parallelogram identity} 
\[
  \| v + w \|^2 + \| v -  w\|^2 = 2\|v\|^2 + 2\|w\|^2 
\]
holds true for all $v,w \in \vectorspV$. In this case, the inner product of two elements $v,w\in \vectorspV$
can be expressed by the \emph{polarization identity for} $\fldK = \R$
\begin{equation}
  \label{eq:real-polarization-identity}
  \langle v , w \rangle = \frac 14 \left( \| v + w \|^2  - \| v - w \|^2 \right)
  = \frac 12 \left( \| v + w \|^2  - \| v \|^2 - \| w \|^2 \right)
\end{equation}
respectively by the  \emph{polarization identity for} $\fldK = \C$
\begin{equation}
  \label{eq:complex-polarization-identity}
  \langle v , w \rangle = \frac 14 \sum_{k=1}^4 \cplxi^k \, \| v + \cplxi^k \, w \|^2  \ .
\end{equation}
\end{theorem}
\begin{proof}
The forward direction is a consequence of
\ref{para:properties-seminorm-associated-positive-semidefinite-hermitian-form}, Eq.~\ref{eq:parallelogram-identity}. 
To show the backward direction we consider two cases $\fldK = \R$ and 
$\fldK = \C$ separately.

\textit{1.~Case.} Given the norm $\|\cdot\|$ define $\langle \cdot , \cdot \rangle : \vectorspV \times \vectorspV \to \R$ by real polarization
\[
   \langle v , w \rangle = \frac 14 \left( \| v + w \|^2  - \| v - w \|^2 \right), \quad \text{where } v,w\in \vectorspV  \ .
\]
Note that the parallelogram identity entails
\[
  \frac 14 \left( \| v + w \|^2  - \| v - w \|^2 \right) =  
  \frac 12 \left( \| v + w \|^2  - \| v \|^2 - \| w \|^2 \right) \ . 
\]
Observe that by definition $\langle v , w \rangle = \langle w , v \rangle$ and $\| v \| = \sqrt{ \langle v , v \rangle}$.
Let us show additivity in the first variable. Let $v_1,v_2,w \in \vectorspV$ and compute using the parallelogram identity 
\begin{equation*}
  \begin{split}
     \| v_1 + v_2 + w \|^2 & = 2 \| v_1 + w \|^2 + 2 \| v_2\|^2 -  \| v_1 + w - v_2\|^2 \ , \\ 
     \| v_1 + v_2 + w \|^2 & = 2 \| v_2 + w \|^2 + 2 \| v_1\|^2 -  \| v_2 + w - v_1\|^2 \ .
  \end{split}
\end{equation*}
Hence 
\begin{equation*}
  \begin{split}
     \| v_1 + v_2 \pm w \|^2 & =  
     \| v_1 \pm w \|^2 +  \| v_2 \pm w \|^2 +  \| v_1\|^2 + \| v_2\|^2 -  \| v_1 \pm w - v_2\|^2  -  \| v_2 \pm w - v_1\|^2 \ .
  \end{split}
\end{equation*}
Subtracting the $-$ version from the $+$ version of this equation entails
\begin{equation*}
  \begin{split}
    \langle v_1 + v_2 , w \rangle \, & = \frac 14 \left( \| v_1 + v_2 + w \|^2  - \| v_1 + v_2 - w \|^2 \right)= \\
    & = \frac 14 \left( \| v_1 + w \|^2 +  \| v_2 + w \|^2 -  \| v_1 - w \|^2 -  \| v_2 - w \|^2 \right) =
    \langle v_1  , w \rangle +  \langle v_2 , w \rangle \ ,
  \end{split}
\end{equation*}
so additivity in the first variable is proved. By induction  one derives from this that 
for all natural $n$
\begin{equation}
\label{eq:natural-number-homogeneity}
  \langle nv , w \rangle = n \langle v , w \rangle \quad \text{for all } v,w \in \vectorspV  \ . 
\end{equation}
Since then $\langle - nv , w \rangle - n \langle v , w \rangle = \langle -nv + nv , w \rangle = 0$ for all $n\in \N$,
Eq.~\eqref{eq:natural-number-homogeneity} also holds for $n\in \Z$. 
Now let $p\in \Z$ and $q \in \gzN$. Then $ q\, \langle \frac pq v , w \rangle = \langle p v , w \rangle = p \, \langle  v , w \rangle$, 
hence one has for rational $r$
\begin{equation}
\label{eq:rational-number-homogeneity}
 \langle rv , w \rangle = r \langle v , w \rangle \quad \text{for all } v,w \in \vectorspV  \ . 
\end{equation}
Since addition, multiplication by scalars and the norm are continuous, the function 
\[
   \R \to \R, \:r \mapsto  \langle rv , w \rangle - r \langle v , w \rangle = \frac 14 \left( \| r v + w \|^2 + r \|  v - w \|^2   - \| rv - w \|^2 -  r \|  v + w \|^2 \right) 
\]
is continuous. Since  it vanishes over $\Q$, it has to coincide with the zero map. Therefore, 
Eq.~\eqref{eq:rational-number-homogeneity} holds for all $r\in \R$. So $\langle \cdot , \cdot \rangle$ is linear in 
the first coordinate. By symmetry, it is so too in the second coordinate. Hence $\langle \cdot , \cdot \rangle$ is a 
symmetric bilinear form inducing $\| \cdot \|$. 

\textit{2.~Case.} In the case $\fldK = \C$ use complex polarization and put
\[
   \langle v , w \rangle = \frac 14 \sum_{k=1}^4 \cplxi^k \, \| v + \cplxi^k \, w \|^2  
   \quad \text{for all } v,w\in \vectorspV  \ .
\]
Then  $\langle \cdot , \cdot \rangle$ is conjugate symmetric, since 
\[
   \overline{\langle v , w \rangle} = \frac 14 \sum_{k=1}^4 (- \cplxi)^k \, \| v + \cplxi^k \, w \|^2  
   =   \frac 14 \sum_{k=1}^4 (-\cplxi)^k \, \| (-\cplxi)^k \, v +  w \|^2 =
   \langle w , v \rangle \ .
\]
Next compute
\[
  \Re \langle v , w \rangle =  \frac 14 \left( \| v + w \|^2  - \| v - w \|^2 \right)
\]
and 
\[
  \Im \langle v , w \rangle =  \frac 14 \left( \| v + \cplxi w \|^2  - \| v - \cplxi w \|^2 \right) \ .
\]
By the first case one concludes that  $\Re \langle \cdot , \cdot \rangle$ and $\Im \langle \cdot , \cdot \rangle$
are both $\R$-linear in the first  and the second coordinate. Moreover,
\[
  \Re \langle \cplxi  v , w \rangle =  \frac 14 \left( \| \cplxi v + w \|^2  - \| \cplxi v - w \|^2 \right)
  = \frac 14 \left( \| v - \cplxi w \|^2  - \| v + \cplxi w \|^2 \right) = 
  - \Im \langle v ,  w \rangle
\]
and 
\[
  \Im \langle \cplxi \, v , w \rangle =  
  \frac 14 \left( \| \cplxi v + \cplxi w \|^2  - \| \cplxi v - \cplxi w \|^2 \right) = 
  \Re \langle  v , w \rangle \ ,
\]
hence $\langle \cdot , \cdot \rangle$ is complex linear in the first coordinate. 
Finally,
\[
  \Re \langle v , v \rangle =  \| v  \|^2 \quad \text{and} \quad 
  \Im \langle v , v \rangle =  \frac 14 \left( \| v + \cplxi v \|^2  - \| v - \cplxi v \|^2 \right) = 0 \ .
\]
This finishes the proof that $ \langle \cdot , \cdot \rangle$ is a complex inner product inducing the 
norm $\| \cdot \|$.
\end{proof}

\para
Next we will turn Hilbert spaces into a category. To this end one needs to know what morphisms in this
category should be. There are two options each giving rise to a category of Hilbert spaces. These categories
just differ by their morphism classes. The first one is to
have as morphisms  linear maps $A:\hilbertH_1\to\hilbertH_2$ preserving the inner products which means that
they fulfill
\[
  \inprod{Av_1,Av_2} = \inprod{v_1,v_2}  \quad \text{for all } v_1,v_2 \in \hilbertH_1 \ .
\] 
By \Cref{thm:parallegram-identity-guarantees-that-norm-comes-from-some-inner-product} this property is
equivalent to
\[
  \| Av \| = \| v \| \quad \text{for all } v \in \hilbertH_1 \ ,
\] 
that is to $A$ being \emph{norm preserving} or \emph{isometric}. Obviously, the identity map between two
Hilbert spaces is isometric and the composition of two composable isometric linear maps between Hilbert
spaces is again isometric and linear. Hence Hilbert spaces together with norm preserving linear maps between
them form a  category which we denote by $\category{Hilb_\textup{np}}$. The isomorphisms in this category
are the surjective and inner product preserving linear maps between Hilbert spaces. Such maps
are called \emph{unitary}. The condition of a linear map being norm preserving is pretty restrictive, so
the category $\category{Hilb_\textup{np}}$ contains only few morphisms. This can be healed by allowing all
\emph{bounded} linear maps between Hilbert spaces to be morphisms that is of all
linear $A:\hilbertH_1\to\hilbertH_2$ for which there exists a $C\geq 0$ such that
\[
  \| Av \| \leq C \| v \| \quad \text{for all } v \in \hilbertH_1 \ .
\]
The smallest such $C$ is called the \emph{operator norm} of $A$ and is denoted $\| A\|$.
Equivalently, the operator norm is given by
\[
  \| A\| = \sup \big\{ \| Av \| \bigmid v\in \hilbertH_1, \, \| v \| \leq 1 \big\}
  = \sup \big\{ \| Av \| \bigmid v\in \hilbertH_1, \, \| v \| = 1 \big\} \ .
\] 
Every norm preserving linear map is bounded with operator norm $1$.
In particular the identity map on a Hilbert space is bounded. Moreover, if
$A: \hilbertH_1\to\hilbertH_2$  and $B: \hilbertH_2\to\hilbertH_3$ are bounded linear operators
between Hilbert spaces, then the composition $BA :  \hilbertH_1\to\hilbertH_3$ is bounded
with operator norm $\leq \| B\|\, \| A\|$ since for all $v \in \hilbertH_1 $ with $\| v\|\leq 1$
\[
  \| BAv \| \leq \| B \| \, \|Av\| \leq \| B\|\, \| A\| \ . 
\]
Hence Hilbert spaces as objects together with bounded linear maps as morphisms form a category which we
denote by $\category{Hilb}$ and call the \emph{category of Hilbert spaces}. Note that the morphisms
in this category appear to ``forget'' the inner product and just preserve the linear and the topological
structure. John Baez \cite[p.~133]{BaeHDAII2HS} has explained how to heal this apparent defect by showing that
$\category{Hilb}$ carries a so-called $*$-structure given by the adjoint map on bounded linear operators.
We will come back to this point later when we introduce adjoint operators. 

\para
Last in this section we will introduce bounded bilinear and sesquilinear maps. We define them for normed
vector spaces. Their main application lies in the operator theory on Hilbert spaces, so we introduce
them here.  

\begin{definition}
  Let $\vectorspV$ be a vector space over $\fldK$ with norm $\| \cdot \| : \vectorspV \to \R_{\geq 0}$. 
  A bilinear or sesquilinear form $b : \vectorspV \times \vectorspV \to \fldK$ is called \emph{bounded} if 
  there exists a $C >0$ such that 
  \[
    | b(v,w)| \leq C \, \| v\| \, \| w \| \quad \text{for all } v,w \in \vectorspV \ .
  \]
  In this case,
  \[
    \| b \| := \sup \big\{ | b(v,w) | \bigmid v,w \in \vectorspV \: \& \: \| v\| = \| w \| = 1 \big\} 
  \] 
  exists and is called the \emph{norm} of the form $b$. 
\end{definition}
\begin{example}
  The inner product on a (pre-) Hilbert space is bounded by the Cauchy--Schwarz inequality and has norm $1$.
\end{example}

\begin{proposition}
A bounded bilinear or sesquilinear form $b : \vectorspV \times \vectorspV \to \fldK$ on a normed vector space $\vectorspV$ over 
$\fldK$ is continuous. Vice versa, if $\vectorspV$ is complete, then continuity of $b : \vectorspV \times \vectorspV \to \fldK$
implies boundedness.
 \end{proposition}
\begin{proof}
If $b$ is bounded, then
\begin{equation*}
  \begin{split} 
  \big\vert b(v,w) - b(v',w') \big\vert \, & 
  \leq   \big\vert b(v,w) - b(v',w) \big\vert + \big\vert b(v',w) - b(v',w') \big\vert \leq \\
  & \leq \| b \| \, \left( \|  w \| \, \| v-v'\| +  \|  v' \| \, \| w-w'\| \right)   
  \end{split}
\end{equation*}
for all $v,v',w,w' \in \vectorspV$. Hence $b$ is locally Lipschitz continuous, so in particular continuous. 

Now assume that $\vectorspV$ is a Banach space and that $b$ is continuous. Then one can find $\delta >0$ such that 
for all $v,w \in  V$ of norm less than $\delta$ the relation $ |b(v,w)| < 1$ holds true. But that entails for all 
non-zero $v,w$
\[
   |b(v,w)| = \frac{4 \, \|v\| \, \|w\|}{\delta^2} \cdot b\left( \delta \frac{v}{2 \|v\|}, \delta \frac{w}{2 \|w\|}\right)
   \leq  \frac{4}{ \delta^2} \|v\| \, \|w\| \ .
\] 
Hence $b$ is bounded.
\end{proof}



