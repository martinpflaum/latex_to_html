% Copyright 2017-2018 Markus J. Pflaum, licensed under CC BY-NC-ND 4.0
% main author: 
%   Markus J. Pflaum
%
\section{Summability}
\label{sec:summability}
\begin{definition}
Assume to be given a locally convex topological vector space $\tvsV$ over the field $\fldK$ of real or complex numbers.
Let $(v_i)_{i \in I}$ be a family of elements of $\tvsV$. Let $\mathscr{F} (I)$ be the
set of finite subsets of $I$ and note that it is filtered by set-theoretic inclusion. 
The family  $(v_i)_{i \in I}$ then gives rise to the net 
$\Big( \sum_{i \in J} v_i \Big)_{J \in \mathscr{F}(I)}$. One calls the family  $(v_i)_{i \in I}$
\emph{summable} to an element $v \in \tvsV$ if the net 
$\Big( \sum_{i \in J} v_i \Big)_{J \in \mathscr{F}(I)}$ converges to $v$. In other words this means that 
for every convex zero neighborhood $U \subset \tvsV$ and $\varepsilon >0$ there exists an element
$J_{U,\varepsilon} \in \mathscr{F} (I)$ such that for all finite sets $J$ with $ J_{U,\varepsilon} \subset J \subset I$
\[
   p_U \left( v - \sum_{i \in J} v_i \right) < \varepsilon \ .
\]
As before, $p_U$ denotes here the gauge of $U$. 
If $\tvsV$ is Hausdorff, the limit $v$ of a summable family  $(v_i)_{i \in I}$ is
uniquely determined, and one  writes in this situation
\[
   v = \sum_{i \in I} v_i \ .  
\]
We denote the space of summable families in $\tvsV$ over the given index set $I$ by 
$\ell^1 (I ,\tvsV)$. For $E=\C$ we just write $\ell^1 (I )$ instead of $\ell^1 (I ,\C)$. If in addition the index set 
coincides with $\N$, we briefly denote $\ell^1 (\N )$ by $\ell^1$.
\end{definition}

\begin{proposition}[Cauchy criterion for summability]
  Let $\tvsV$ be a complete locally convex topological vector space. 
  A family $(v_i)_{i \in I}$ of elements of $\tvsV$ then is summable 
  to some $v\in \tvsV$ if and only if it satisfies the following Cauchy condition:
  \begin{axiomlist}[\hspace{1pt}]
  \item[\textup{\sffamily (C)}] 
    For every convex zero neighborhood $U \subset \tvsV$ and $\varepsilon >0$ there exists an element
    $J_{U,\varepsilon} \in \mathscr{F} (I)$ such that for all $K \in \mathscr{F} (I)$ with  $ K \cap J_{U,\varepsilon} = \emptyset$
    the relation 
    \[
      p_U \left( \sum_{i \in K} v_i \right) < \varepsilon 
    \]
    holds true. 
  \end{axiomlist}
 
\end{proposition}
\begin{proof}
  By completeness of $\tvsV$ it suffices to verify that the net $\Big( \sum_{i \in J} v_i \Big)_{J \in \mathscr{F}(I)}$ is a Cauchy net
  if and only if condition \textup{\sffamily (C)} is satisfied. Recall that one calls
  $\Big( \sum_{i \in J} v_i \Big)_{J \in \mathscr{F}(I)}$ a Cauchy net if for every convex zero neighborhood $U \subset \tvsV$ 
  all $\varepsilon >0$ there exists an element $J_{U,\varepsilon} \in \mathscr{F} (I)$ such that for all $J,J' \in \mathscr{F} (I)$ 
  containing  $ J_{U,\varepsilon}$ as a subset the relation 
  \[
     p_U \left( \sum_{i \in J} v_i - \sum_{i\in J'} v_i \right) < \varepsilon 
  \]
  holds true. 
  But that is clearly equivalent to condition \textup{\sffamily (C)}. 
\end{proof}

\para Several other notions of summability have been introduced in the analysis and functional analysis literature.  
These are mainly either used to establish summability criteria or are used in the study of 
topological tensor products and nuclearity of locally convex topological vector spaces, see \cite{GroPTTEN,PieNLCS}. 
In the following we define these further notions of summability and study their properties.
The symbol $\tvsV$  hereby always stands for a locally convex \tvs, $I$ always denotes a nonempty
index set, and $\mathscr{F} (I)$ the set of its finite subsets.

\begin{definition}
  A family   $(v_i)_{i \in I}$ in $\tvsV$ is called \emph{weakly summable} to $v \in \tvsV$ if for 
  every continuous linear form $\alpha : \tvsV \to \fldK$ the net 
  $\Big( \sum_{i \in J} \alpha( v_i ) \Big)_{J \in \mathscr{F}(I)}$ converges in $\fldK$ to $\alpha ( v )$.
  In other words this means that for every $\alpha \in \tvsV'$ and $\varepsilon >0$ there exists a finite set 
  $J_{\alpha,\varepsilon} \subset I$ such that for all finite sets $J$ with $J_{\alpha,\varepsilon} \subset  J \subset I $ 
  \[
    \left| \alpha (v) -  \sum_{j\in J} \alpha( v_i )   \right|< \varepsilon \ . 
  \]
  The set of all weakly summable families in $\tvsV$ with index set $I$ is denoted $\ell^1 [I, \tvsV]$.
\end{definition}

\begin{definition}
  A family  $(v_i)_{i \in I}$ in $\tvsV$ is called \emph{absolutely summable} if for every circled convex 
  zero neighborhood $U \subset \tvsV$ there exists some $C\geq 0$ such that
  \[
    \sum_{i \in J} p_U \left( v_i\right) \leq C \quad \text{for all } J \in \mathscr{F} (I) \ .
  \]
  We denote the set of all absolutely summable families  in $\tvsV$ by  $\ell^1 \{ I, \tvsV \}$.
\end{definition}

\begin{proposition}
  A family $(v_i)_{i \in I} \subset \tvsV$ is absolutely summable if and only if for every element $U$ 
  of a basis of circled convex zero neighborhoods there exists a $C\geq 0$ such that
  \[
    \sum_{i \in J} p_U \left( v_i\right) \leq C \quad \text{for all } J \in \mathscr{F} (I) \ .
  \]
\end{proposition}

\begin{proof}
  
\end{proof}

\begin{definition}
 A family $(v_i)_{i \in I}$ in $\tvsV$ is called \emph{totally summable} if there exists a bounded 
 absolutely convex subset $B\subset \tvsV$  and a $C\geq 0$ such that
 \[
   \sum_{i \in J} p_B \left( v_i\right) \leq C  \quad \text{for all } J \in \mathscr{F} (I) \ .
 \]
 We write $\ell^1 \langle I, \tvsV \rangle$ for the set of all totally summable families in $\tvsV$.
\end{definition}

\subsec{Summable families of complex numbers}

\begin{lemma}[cf.~{\cite[Lem.~1.1.2]{PieNLCS}}]
  Let $(z_i)_{i\in I}$ be a family of  complex numbers for which there exists a positive real number $C > 0$ 
  such that 
  \[
     \left|\sum_{i\in J} z_i \right| \leq C \quad \text{for all } J \in \mathscr{F} (I) \ .
  \]
  Then one has the estimate
   \[
     \sum_{i\in J}\left| z_i \right| \leq 4 C \quad \text{for all } J \in \mathscr{F} (I) \ .
  \]
\end{lemma}

\begin{proof}
  We assume first that all $z_i$ are real. Then let $I^+$ the set of all indices $i\in I$ such that $z_i \geq 0$,
  and $I^-$ the set of all  $i\in I$ such that $z_i < 0$. Then, for all finite $J\subset I$
  \[
     \sum_{i\in J} \left| z_i \right| =  \sum_{i\in J\cap I^+}  \left| z_i  \right| +  \sum_{i\in J\cap I^-}  \left| z_i  \right|
    =  \left| \sum_{i\in J\cap I^+}  z_i  \right| +  \left| \sum_{i\in J\cap I^-}  z_i  \right| \leq 2  C \ .
  \] 
  In the general case decompose $z_i$ into real and imaginary parts $x_i = \Re z_i$ and   $y_i = \Im z_i$.
  By the triangle inequality one obtains for all finite $J\subset I$
  \[
    \sum_{i\in J} \left| z_i \right| \leq \sum_{i\in J} \left| x_i \right| + \sum_{i\in J} \left| y_i \right| \leq 4 C \ .
  \]
\end{proof}

\begin{proposition}
\label{thm:summability-criteria-family-complex-numbers}
  For a  family $(z_i)_{i\in I}$ of complex numbers the following are equivalent.
  \begin{romanlist}
  \item The family $(z_i)_{i\in I}$ is summable. 
  \item The family $(\left| z_i \right| )_{i\in I}$ is summable.
  \item The family $(z_i)_{i\in I}$ is absolutely summable.

  \item There exists some $C > 0$ such that 
        $\sum_{i\in J}\left| z_i \right| \leq C$ for all $J\in \mathscr{F} (I)$.
  \end{romanlist}
  In case that  one hence all of the conditions are fulfilled, the estimate 
  \[
     \left|  \sum_{i\in I} z_i \right|  \leq \sum_{i\in I}\left| z_i \right| 
  \]
  holds true.
\end{proposition}

\begin{proof}
  Assume that  $(z_i)_{i\in I}$ is absolutely summable. Since $\C$ is normed with norm given by the absolut value 
  this just means that there exists some $C > 0$ such that 
  $\sum_{i\in J}\left| z_i \right| \leq C$ for all $J\in \mathscr{F} (I)$. Hence the supremum 
  $c = \sup \left\{ \sum_{i\in J}\left| z_i \right| \mid J \in \mathscr{F} (I)\right\}$ exists and is $\leq C$. 
  For given $\varepsilon >0$ choose $J_\varepsilon \in \mathscr{F} (I)$ such that 
  \[
        c- \varepsilon \leq \sum_{i\in J_\varepsilon}\left| z_i \right| \leq c \ .
  \]
  Then one has for all $K\in \mathscr{F} (I)$ with $K\cap J_\varepsilon = \emptyset$ 
  \[
     \left| \sum_{i\in K} z_i \right| \leq  \sum_{i\in K} \left| z_i \right| \leq \varepsilon \ .
  \]
  Hence $\left( \sum_{i\in J} z_i \right)_{J \in \mathscr{F} (I)}$ is a Cauchy net, so has to
  converges  by completeness of $\C$. This proves summability of   $(z_i)_{i\in I}$.
  
  Vice versa, assume now that   $(z_i)_{i\in I}$ is summable. Then
  $\left( \sum_{i\in J} z_i \right)_{J \in \mathscr{F} (I)}$ is a Cauchy net. Hence there exists an element 
  $J_1 \in  \mathscr{F} (I)$ such that for all $K\in  \mathscr{F} (I)$ with $K \cap J_1 = \emptyset$  
  the inequality 
  \[
      \left| \sum_{i\in K} z_i \right| < 1 
  \]
  holds true. Let $C = \sum_{i\in J_1} \left| z_i\right|$. Then one has for all $J\in  \mathscr{F} (I)$
  \[
       \left| \sum_{i\in J} z_i \right| \leq  \left| \sum_{i\in J\setminus J_1} z_i \right| +  
       \left| \sum_{i\in J\cap J_1} z_i \right| \leq 1 + C \ .
  \]
  By the preceding lemma the set of partial sums $ \sum_{i\in J} \left| z_i \right|$, where $J$ runs through the 
  finite subsets of $I$, is then bounded by $4 + 4C$, hence  $(z_i)_{i\in I}$ is absolutely summable.
\end{proof}

\subsec{Summability in Banach spaces}

\begin{proposition}\label{thm:summability-criteria-family--normed-vector-space}
  Let $\banachV$ be a normed vector space. 
  For a  family $(v_i)_{i\in I}$ of elements in $\tvsV$ the following are equivalent:
  \begin{romanlist}
  \item\label{ite:absolute-summable-family-normed-vector-space} The family $(v_i)_{i\in I}$ is absolutely summable. 
  \item\label{ite:norm-summable-family-normed-vector-space} The family $(\left\| v_i \right\| )_{i\in I}$ is summable.
  \item\label{ite:cauchy-summable-family-normed-vector-space} There exists some $C > 0$ such that 
        $\sum_{i\in J}\left\| v_i \right\| \leq C$ for all $J\in \mathscr{F} (I)$.
  \end{romanlist}
  If $\banachV$ is even a Banach space, these conditions are all equivalent to 
  \begin{romanlist}
  \setcounter{enumi}{3}
  \item The family $(v_i)_{i\in I}$ is summable.
  \end{romanlist}
\end{proposition}
\begin{proof}
  \ref{ite:norm-summable-family-normed-vector-space} and \ref{ite:cauchy-summable-family-normed-vector-space}
  are equivalent by \Cref{thm:summability-criteria-family-complex-numbers}
  Assume now that \ref{ite:absolute-summable-family-normed-vector-space} holds true. 
\end{proof}


\textbf{to do:} Carl Neumann series


\subsec{Properties of and relations between the various summability types} 

\begin{theorem}
  Let $I$ be a non-empty index set. Then the spaces $\ell^1 (I, \tvsV )$ of summable families, 
  $\ell^1 [I, \tvsV ]$ of weakly summable families, $\ell^1 \{I, \tvsV \}$ of absolutely summable families and
  $\ell^1 \langle I, \tvsV \rangle$ of totally summable families in $E$ are all subvector spaces of 
  the product vector space $E^I = \Pi_{i\in I} E$.   Furthermore one has the following chain of inclusions:
  \[
     \ell^1 \langle I, \tvsV \rangle\subset\ell^1 \{I, \tvsV \}   \quad\text{and} \quad \ell^1 (I, \tvsV) \subset \ell^1 [I, \tvsV ] \ .
  \] 
  If $E$ is complete, then one even has 
   \[
        \ell^1 \{I, \tvsV \}   \subset \ell^1 (I, \tvsV)     
   \]
\end{theorem}

\begin{proof}
    Now let $(v_i)$ be a summable family and $\alpha : \tvsV \to \fldK$ a continuous linear form. 

 Let $U$ be an absolutely convex zero neighborhood. Then $U$ absorbes $B$, so there exists $r>0$ such that
 $B \subset rU$. Hence
\end{proof}


