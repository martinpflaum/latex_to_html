% Copyright 2017 Markus J. Pflaum, licensed under CC BY-NC-ND 4.0
% main author: 
%   Markus J. Pflaum
%
\section{Orthogonal decomposition and the Riesz representation theorem}
  
\para 
One of the issues with infinite-dimensional analysis is that a closed subspace of an infinite dimensional Banach space might not have a 
closed complement. Fortunately, the situation in Hilbert space theory is not so grim because 
every closed subspace of a Hilbert space admits an orthogonal complement. This is one of the four 
crucial properties  which distinguish Hilbert spaces from Banach spaces and which are stated in the following. 

In this section $\hilbertH$ will always denote a Hilbert space over the field $\fldK=\R$ or $\fldK=\C$. 
The symbol $\langle \cdot , \cdot\rangle$ will stand for the inner product of $\hilbertH$. 

\begin{theorem}[Best approximation theorem] 
   Every closed convex nonempty subset $C$ of a Hilbert space 
   $\hilbertH$ has a unique element of minimal norm.
\end{theorem}
\begin{proof}
Let $d = \inf\{ \|v\| \mid v \in C \}$ which is a non-negative real number. We claim there exists a unique 
$v_0 \in C$ with $\|v_0\| =d$. 
For uniqueness, consider two vectors $v_0, v_1$ satisfying the desired property, and let $v = \frac{1}{2}(v_0 + v_1)$ 
be their midpoint. Then
\begin{equation*}
  \|v\| = \frac{1}{2}\|v_0 + v_1\| \leq \frac{1}{2}(\|v_0\| + \|v_1\|) = d
\end{equation*}
By minimality of $d$ this entails $\|v\| =d$. By the parallelogram identity
\begin{equation*}
    \left\|\frac{1}{2} (v_0 + v_1)\right\|^2 + \left\|\frac{1}{2}(v_0 - v_1) \right\|^2  = 
    2\left\|\frac{v_0}{2}\right\|^2 + 2\left\|\frac{v_1}{2}\right\|^2 = d^2 \ ,
\end{equation*}
hence
\[
\left\| \frac{1}{2} (v_0 - v_1) \right\|^2 \leq d^2 - \|v\|^2 = 0 \ ,
\]
proving $v_0 = v_1$. 

For the proof of existence observe that by definition of $d$ there exists a sequence $(v_n)_{n \in \mathbb{N}} \subset C$ such 
that $\lim_{n \to \infty}\|v_n \| = d$. By convexity
\[
   \frac{1}{2}(v_n + v_m) \in C
\]
for all $n,m \in \mathbb{N}$, hence $\frac{1}{4}\|v_n + v_m \|^2 \geq d^2$. The parallelogram equality entails
\[
0 \leq \| v_n - v_m \|^2 = 2\|v_n \|^2 + 2\|v_m \|^2 - \| v_n + v_m \|^2 \leq 2\|v_n \|^2 + 2\|v_m \|^2 - 4d^2 \ .
\]
Since $\lim_{n \to \infty}\|v_n \| = d$ there exists for given $\varepsilon >0$ an $N\in \N$ such that 
$ \|v_n \|^2 -d^2 \leq \frac 14 \varepsilon^2 $ for all $n \geq N$. Hence, for $n,m \geq N$
\[
  0 \leq \| v_n - v_m \| \leq \varepsilon \ ,
\] 
and $(v_n)_{n \in \N}$ is a Cauchy-sequence, so convergent by completeness of $\hilbertH$. Put $v_0:= \lim_{n \to \infty}v_n$. 
Then $v_0 \in C$ since $C$ is closed and $\|v_0 \|=\lim_{n \to \infty}\|v_n \| = d$. The existence claim follows 
and the proof is finished.
\end{proof}

\begin{thmanddef}[Orthogonal decomposition theorem] 
\label{thm:orthogonal-decomposition-theorem}
Let $\vectorspV \subset \hilbertH$ be a closed subspace of the Hilbert space $\hilbertH$. Then the 
\emph{orthogonal complement}
\[
  \vectorspV^\bot = \big\{ w \in \hilbertH \bigmid \langle v,w \rangle = 0 \text{ for each } v \in \vectorspV \big\}
\]
is a closed subspace of $\hilbertH$ and $\hilbertH = \vectorspV \oplus \vectorspV^\bot$. 
The map $\proj_\vectorspV : \hilbertH \to \vectorspV$ which maps $w \in \hilbertH$ to the unique $w_1\in \vectorspV$ 
such that $w - w_1 \in   \vectorspV^\bot$ is called the \emph{orthogonal projection} onto $\vectorspV$.
It satisfies $\left\| w-  \proj_\vectorspV (w)\right\| = d(w,\vectorspV) := \inf \big\{ \| v-w \| \bigmid v\in \vectorspV \big\}$ that is $\proj_\vectorspV (w)$ is the unique element of $\vectorspV$ having shortest distance from $w$.  
\end{thmanddef}
\begin{proof}
For $v \in \hilbertH$ define $v^\flat :\hilbertH \to \R$ by $v^\flat (w) = \langle w,v \rangle$. Recall that this map is 
continuous and linear. Hence the kernel $(v^\flat)^{-1}(0)$ is a closed linear subspace of $\hilbertH$ and 
\[
  \vectorspV^{\bot} = \bigcap_{v \in \vectorspV} (v^\flat)^{-1}(0)
\]
is a closed linear subspace. To show $\vectorspV \cap \vectorspV^\bot = \{0\}$, consider 
$v \in \vectorspV \cap \vectorspV^\bot$. Then $\|v \|^2 = \langle v,v \rangle = 0$. 
Now, given some $w \in \hilbertH$, it can be written as $w = w_1 + w_2$ with 
$w_1 \in \vectorspV$ and $w_2 \in \vectorspV^\bot$. To see this put $C = w - \vectorspV$. Then $C$ is closed and convex. 
By the best approximation theorem there exists a unique element $w_2 \in C$ of minimal norm. Let  $w_1$
be the unique element of $\vectorspV$ such that $w_2 = w -w_1$. It remains to show $w_2 \in \vectorspV^\bot$. 
Since $w_2$ has minimal norm among the elements of $w-\vectorspV$ the following inequality holds for all 
vectors $v \in \vectorspV$:
\[
\|w_2 \|^2 \leq \| w_2 + v \|^2 = \|w_2 \|^2 + 2\, \Re\langle w_2,v\rangle + \| v \|^2 \ . 
\]
Hence
\[
0 \leq 2\, \Re\langle w_2,v\rangle + \| v \|^2 \quad \text{for all  } v \in \vectorspV  \ .
\]
Now assume that $\|v\|=1$ and choose $\varphi \in \R$ such that $e^{i\varphi}\langle w_2, v \rangle \in \R$.
Setting $v' = e^{-i\varphi}v$, one obtains for all $\lambda \in \R$ by the last inequality
\[
  0 \leq 2 \langle w_2,\lambda v'\rangle + \| \lambda v' \|^2 = 
  2\lambda\langle w_2, x'\rangle + \lambda^2\ .
\]
For $\lambda = -\langle w_2, v' \rangle$ this entails  the estimate
\[ 
   \| \langle w_2, v' \rangle \|^2 = 
   - \left( - 2\|\langle w_2,v \rangle \|^2 + \| \langle w_2, v' \rangle \|^2 \right) = - \left(  2\lambda\langle w_2, x'\rangle + \lambda^2 \right) \leq 0 \ .
\]
Hence $\langle w_2, v \rangle = 0$ for all unit vectors $v \in \vectorspV$, therefore $w_2 \in \vectorspV^\bot$.

The remainder of the claim is now a consequence of the construction of $w_1$ from the given $w$ and the 
observation that $\proj_\vectorspV (w) = w_1$. 
\end{proof}

\begin{corollary}
  For every closed subspace $\vectorspV \subset \hilbertH$ of a Hilbert space $\hilbertH$ the relation
  \[
     V = (V^\perp)^\perp 
  \]
  holds true. 
\end{corollary}

\begin{proof}
  One has $V \subset  (V^\perp)^\perp $ by definition of the orthogonal complement. 
  Since \[ \hilbertH = V \oplus  V^\perp =  (V^\perp)^\perp \oplus V^\perp \] 
  by the preceding theorem, the claim follows.
\end{proof}

\begin{theorem}[Riesz representation theorem for Hilbert spaces]
  Let $\hilbertH$ be a Hilbert space and $\hilbertH'$ its topological dual. Then the \emph{musical map} 
  \[ {}^\flat: \hilbertH \to \hilbertH',\quad 
     v \mapsto v^\flat = \left( \hilbertH \ni w \mapsto \langle w,v\rangle \in \fldK\right) \]
  is an isometric isomorphism which is linear in the real case and conjugate-linear in the complex case. 
\end{theorem}

\begin{proof}
  Obviously, ${}^\flat$ is linear if the ground field $\fldK$ equals $\R$ and conjugate-linear if $\fldK=\C$.
  Now observe that for all $v \in \hilbertH$ by the Cauchy--Schwarz inequality 
  \[
    \| v^\flat \| = \sup\big\{ |\langle w,v \rangle | \bigmid w \in \hilbertH \: \& \: \|w\| =1 \big\} = 
    \| v \| \ ,
  \]
  hence ${}^\flat$ is an isometry, so in particular injective. 
  It remains to show surjectivity. So assume that $\alpha : \hilbertH \to \fldK$ is a nontrivial 
  continuous linear form. 
  Let $\vectorspV$ be its kernel. Then  $\vectorspV$ is a closed linear subspace of $\hilbertH$.
  Since $\alpha$ is nontrivial, the orthogonal complement $\vectorspV^\bot$ is nontrivial, too. Hence 
  $\vectorspV^\bot \cong \hilbertH/\vectorspV$ is isomorphic to $\image \alpha = \fldK$ 
  and there exists a  vector $v \in \vectorspV^\bot \setminus \{ 0 \} $ such that 
  $\alpha (v) = 1$. Since $v$ spans $ \vectorspV^\bot$ there exists   for every $w\in \hilbertH$ a
  unique $\lambda_w  \in \fldK$ such that $w = \proj_V (w) + \lambda_w v$. Then compute 
  \[
    \alpha (w) = \alpha (\lambda_w v ) = \lambda_w  \quad \text{and} \quad 
    \left( \frac{v}{\|v\|^2}\right)^\flat (w) = 
    \frac{1}{\|v\|^2} \langle w, v \rangle =  \frac{ \lambda_w}{\|v\|^2} \langle v , v \rangle 
    = \lambda_w \ .
  \]
  This entails $\alpha = \left( \frac{v}{\|v\|^2}\right)^\flat$, and ${}^\flat$ is surjective.
\end{proof}

\begin{remark}
  Sometimes, and we will follow that convention, the inverse of the musical isomorphism 
  ${}^\flat: \hilbertH \to \hilbertH'$ is denoted ${}^\sharp: \hilbertH' \to \hilbertH$. 
\end{remark}

\begin{corollary}
  Every Hilbert space $\hilbertH$ is \emph{reflexive} that is the canonical map
  \[ H \to H'' , \: v \mapsto \left( H' \ni \lambda \mapsto \lambda(v) \in \fldK \right) \]
  is an isometric isomorphism. 
\end{corollary}

\begin{proof}
  By the Riesz Representation Theorem, the dual $\hilbertH'$ is a Hilbert space 
  with inner product  
  \[
    \langle\!\langle \cdot, \cdot  \rangle\!\rangle : \hilbertH' \times \hilbertH' \to \fldK, \:  
    (\lambda,\mu)   \mapsto \langle\!\langle \lambda, \mu \rangle\!\rangle = \inprod{\mu^\sharp,\lambda^\sharp} \ .
  \] 
  Hence, by applying the Riesz Representation Theorem twice, 
  the map ${}^\flat \circ {}^\flat : \hilbertH \to  \hilbertH''$ is an isometric linear isomorphism. 
  Now compute for $v\in \hilbertH$ and $\lambda \in \hilbertH'$
  \[
     (v^\flat)^\flat (\mu) = \langle\!\langle \lambda, v^\flat \rangle\!\rangle =
     \inprod{v, \lambda^\sharp} = \lambda (v) \ . 
  \]
  Hence ${}^\flat \circ {}^\flat$ coincides with the canonical map above and the claim
  follows. 
\end{proof}

\begin{corollary}
\label{thm:correspondence-bounded-sesquilinear-forms-bounded-operators}
  Let $b:\hilbertH \times \hilbertH\to \fldK$ be a bounded sesquilinear form on a Hilbert space $\hilbertH$. Then there exists
  unique bounded linear map $A : \hilbertH \to \hilbertH$ such that
  \[
      b(v,w)  = \inprod{Av,w} \quad \text{for all } v,w\in \hilbertH \ .
  \]
  Moreover, the operator norm $\| A \|$ coincides with $\| b\|$. 
\end{corollary}
\begin{proof}
  First let us show uniqueness. So let $A,B  : \hilbertH \to \hilbertH$ be bounded and linear so that
  \[
          b(v,w)  = \inprod{Av,w} = \inprod{Bv,w}  \quad \text{for all } v,w\in \hilbertH \ .
  \]
  Then $ \|(A-B)v \|^2 = \inprod{Av-Bv,(A-B)v} = b(v,(A-B)v) - b(v,(A-B)v)=  0$ for all $v\in \hilbertH$ 
  which entails equality of $A$ and $B$. 

  To prove existence observe that for every $v\in \hilbertH$ the map $\hilbertH \to \fldK$, $w \mapsto \overline{b(v,w)}$
  is bounded an linear, so by the Riesz representation theorem there exists for every $v$ an element $Av \in \hilbertH$ 
  such that $\inprod{w,Av} =  \overline{b(v,w)}$ for all $w\in \hilbertH$. 
  Let us show that the map $A$ is linear. 
  For $v_1,v_2\in \hilbertH$ check that
  \begin{equation*}
    \begin{split}
      \inprod{w,A(v_1+v_2)} & =  \overline{b(v_1+v_2,w)} = \overline{b(v_1,w)}+ \overline{b(v_2,w)} = \\
      & = \inprod{w,Av_1}  +\inprod{w,Av_2} =  \inprod{w,Av_1+Av_2} \quad \text{for all } w\in \hilbertH \ . 
    \end{split}
  \end{equation*}
  But that implies $A(v_1+v_2) = Av_1+Av_2$. Given $r\in \fldK$ and $v\in\hilbertH$ one verifies
  \[
    \inprod{w,A(rv)} =  \overline{b(rv,w)} = \overline{rb(v,w)} = \overline{r} \, \overline{b(v,w)}
    =  \overline{r}  \inprod{w,Av} =  \inprod{w,rAv} \quad \text{for all } w\in \hilbertH \ . 
  \]
  Hence $A(rv) = r Av$ and linearity of $A$ is proved. 
  
  For the operator norm compute
  \begin{equation*}
    \begin{split}
    \| A \| & = \sup \big\{ \left|\inprod{Av,w} \right| \bigmid v,w \in \hilbertH \: \& \: \|v\| = \|w\| =1 \big\} = \\
    & = \sup \big\{ \left| b(v,w) \right| \bigmid v,w \in \hilbertH \: \& \: \|v\| = \|w\| =1 \big\} = \| b\| \ .
   \end{split}
  \end{equation*}
\end{proof}

\para
Last in this section we will examine the \emph{Hilbert direct sum} or just \emph{Hilbert sum} of a family
$(\hilbertH_i)_{i\in I}$ of Hilbert spaces. It is defined by
\begin{equation*}
  \begin{split}
   \widehat{\bigoplus\limits_{i\in I}} \hilbertH_i & = \left\{ (v_i)_{i\in I} \in \prod_{i\in I}\hilbertH_i
   \Bigmid \left( \|v_i\|^2 \right)_{i\in I} \text{ is summable} \right\} = \\
   & = \left\{ (v_i)_{i\in I} \in \prod_{i\in I}\hilbertH_i
   \Bigmid \exists C \geq 0 \, \forall J \in \mathscr{F} (I): \: \sum_{i\in J} \|v_i\|^2 \leq C \right\} \ ,
  \end{split}
\end{equation*}
where, as usual, $\mathscr{F} (I)\subset \powerset{I}$ denotes the set of all finite subsets of $I$.

\begin{proposition}
  Let $(\hilbertH_i)_{i\in I}$ be a family of Hilbert spaces. Then the Hilbert direct sum
  $\widehat{\bigoplus\limits_{i\in I}} \hilbertH_i$ is a Hilbert space with inner product  given by
  \[
    \inprod{-,-} : \widehat{\bigoplus\limits_{i\in I}} \hilbertH_i \times
    \widehat{\bigoplus\limits_{i\in I}} \hilbertH_i \to \fldK, \quad
    \left( (v_i)_{i\in I} ,  (w_i)_{i\in I} \right) \mapsto \sum_{i\in I} \inprod{v_i,w_i} \ .
  \]
\end{proposition}

\begin{proof}
  We show first that $\widehat{\bigoplus\limits_{i\in I}} \hilbertH_i$ is a subvector space of
  the direct product $\prod_{i\in I} \hilbertH_i$.  
  Let $z\in \fldK$ and $(v_i)_{i\in I}, (w_i)_{i\in I}\in \widehat{\bigoplus\limits_{i\in I}} \hilbertH_i$.
  Choose $C,D \geq 0$ such that
  \[
    \sum_{i\in J} \|v_i\|^2 \leq C \quad\text{and}\quad
    \sum_{i\in J} \|w_i\|^2 \leq D \quad\text{for all} J \in \mathscr{I} \ .
  \]
  Then
  \begin{equation}
    \label{eq:estimate-finite-sum-square-norms-multiple}
    \sum_{i\in J} \|z v_i\|^2 = |z| \, \sum_{i\in J} \| v_i\|^2\leq |z| \, C \quad
    \text{for all } J\in \mathscr{I} \ ,
  \end{equation}
  so $(zv_i)_{i\in I} \in \widehat{\bigoplus\limits_{i\in I}} \hilbertH_i$. 
  Moreover, by Minkowski's inequality for finite sums,
  \begin{equation}
    \label{eq:estimate-finite-sum-square-norms-sum}
    \sum_{i\in J} \|v_i + w_i\|^2 \leq
    \left( \sqrt{\sum_{i\in J} \|v_i\|^2} +  \sqrt{\sum_{i\in J} \| w_i\|^2} \right)^2
    \leq \left( \sqrt{C} + \sqrt{D} \right)^2 \quad\text{for all } J\in \mathscr{I} \ .
  \end{equation}
  Hence the family $\left( \| v_i + w_i\|^2 \right)_{i\in I}$ is summable and
  $\left( v_i + w_i \right)_{i\in I} \in \widehat{\bigoplus\limits_{i\in I}} \hilbertH_i$. 
  
  Next observe that the map
  \[
    \big\| - \big\| : \widehat{\bigoplus\limits_{i\in I}} \hilbertH_i \to \fldK,
    \: (v_i)_{i\in I} \mapsto \big\| (v_i)_{i\in I}  \big\| = \sqrt{\sum_{i\in I} \| v_i\|^2} 
  \]
  is well-defined by definition of the Hilbert direct sum. It is even a norm
  by \eqref{eq:estimate-finite-sum-square-norms-multiple} and
  \eqref{eq:estimate-finite-sum-square-norms-sum}. 
    
  Now we need to show that the inner product on $\widehat{\bigoplus\limits_{i\in I}} \hilbertH_i$
  is well-defined which means that the family $\left( \inprod{v_i,w_i} \right)_{i\in I}$ is summable
  for all $(v_i)_{i\in I} ,  (w_i)_{i\in I} \in \widehat{\bigoplus\limits_{i\in I}} \hilbertH_i$.
  To this end let $J\subset I$ be a finite subset. Then, by the triangle inequality, 
  the Cauchy--Schwarz inequality on the Hilbert spaces $\hilbertH_i$ and the
  Cauchy--Schwarz inequality for finite sums,
  \[
    \left| \sum_{i\in J} \inprod{v_i,w_i} \right| \leq
    \sum_{i\in J} \left| \inprod{v_i,w_i} \right| \leq
    \sum_{i\in J} \| v_i\| \, \| w_i\| \leq
    \sqrt{\sum_{i\in J} \| v_i\|^2} \cdot \sqrt{\sum_{i\in J} \| w_i\|^2}
    \leq \big\| (v_i)_{i\in I}  \big\| \, \big\| (w_i)_{i\in I}  \big\| \ .
  \]
  Hence the family $\left( \inprod{v_i,w_i} \right)_{i\in I}$ is absolutely summable, so in particular
  summable, and the inner product is well-defined.

  By definition and since all the inner products on the Hilbert spaces $\hilbertH_i$ are conjugate
  symmetric and positive definite, the map $\inprod{-,-} $ on
  $\widehat{\bigoplus\limits_{i\in I}} \hilbertH_i$ has to be conjugate symmetric and positive definite
  as well. It remains to show linearity in the first argument.
  Denote for  $(v_i)_{i\in I}, (w_i)_{i\in I} \in \prod_{i\in I} \hilbertH_i$
  and $J\in \mathscr{F} (I)$ by $\inprod{(v_i)_{i\in I} ,(w_i)_{i\in I}}_J$ the finite sum
  $\sum_{i\in J} \inprod{v_i,w_i}$. Observe that the net
  $\big( \inprod{(v_i)_{i\in I} ,(w_i)_{i\in I}}_J \big)_{J\in \mathscr{F}(I)}$ converges to
  $\inprod{(v_i)_{i\in I} ,(w_i)_{i\in I}}$ in case both $(v_i)_{i\in I}$ and $(w_i)_{i\in I}$
  are in $\widehat{\bigoplus\limits_{i\in I}} \hilbertH_i$. Now let $z\in \fldK$ and 
  $(v_i)_{i\in I}, (v_i^\prime)_{i\in I},  (w_i)_{i\in I} \in \widehat{\bigoplus\limits_{i\in I}} \hilbertH_i$.
  Then 
  \begin{equation*}
    \begin{split}
      \inprod{(v_i)_{i\in I} + (v_i^\prime)_{i\in I},(w_i)_{i\in I}}_J & =
      \inprod{(v_i)_{i\in I} ,(w_i)_{i\in I}}_J + \inprod{(v_i^\prime)_{i\in I},(w_i)_{i\in I}}_J
      \quad\text{and}\\
      \inprod{z (v_i)_{i\in I},(w_i)_{i\in I}}_J & = z \inprod{(v_i)_{i\in I} ,(w_i)_{i\in I}}_J \ .
    \end{split}
  \end{equation*}
  By convergence of all the nets $\big( \inprod{(v_i)_{i\in I} ,(w_i)_{i\in I}}_J \big)_{J\in \mathscr{F}(I)}$,
  linearity in the first argument follows.

  By construction, the norm associated to the inner product $\inprod{-,-} $ on
  $\widehat{\bigoplus\limits_{i\in I}} \hilbertH_i$ coincides with the above defined norm $\big\|-\big\|$.
  It remains to show that $\widehat{\bigoplus\limits_{i\in I}} \hilbertH_i$ equipped with the norm
  $\big\|-\big\|$ is complete. To this end observe that for every finite $J\subset I$ the map
  \[
    \big\| - \big\|_J : \prod_{i\in I} \hilbertH_i \to \R_{\geq 0}, \:
    (v_i)_{i\in I} \mapsto \sqrt{\inprod{(v_i)_{i\in I},(v_i)_{i\in I}}_J}
    = \sqrt{\sum_{i\in J} \| v_i \|^2}
  \]
  is a seminorm and that $(v_i)_{i\in I} \in \prod_{i\in I} \hilbertH_i$ lies in
  the Hilbert direct sum $\widehat{\bigoplus\limits_{i\in I}} \hilbertH_i$ if and only if the
  family $\left(\big\| (v_i)_{i\in I}  \big\|_J\right)_{J\in \mathscr{F}(I)}$ is bounded. 
  Now let $\left((v_i^n)_{i\in I} \right)_{n\in \N}$ be a Cauchy sequence.
  Let $\varepsilon >0$ and choose $N_\varepsilon \in \N$ such that 
  \begin{equation}
    \label{eq:cauchy-criterion-sequence-hilbert-direct-sum}
    \big\| (v_i^m)_{i\in I} - (v_i^n)_{i\in I} \big\| < \varepsilon \quad
    \text{for all } n,m\geq N_\varepsilon \ .
  \end{equation}
  Hence
  \begin{equation}
    \label{eq:cauchy-criterion-sequence-finite-cutoff}
    \big\| (v_i^m)_{i\in I} - (v_i^n)_{i\in I} \big\|_J <    \varepsilon \quad
    \text{for all } J\in \mathscr{F}(I) \text{ and }  n,m\geq N_\varepsilon \ . 
  \end{equation}
  Taking $J =\{ j\}$ for $j\in I$ this implies that the sequence
  $ (v_j^n)_{n\in \N}$ is a Cauchy sequence in the Hilbert space $\hilbertH_j$. 
  Let $v_j \in \hilbertH_j$ be its limit. The family $(v_i)_{i\in I}$ then is
  an element of $\widehat{\bigoplus\limits_{i\in I}} \hilbertH_i$. To verify this
  put $N=N_1$ and observe that by \eqref{eq:cauchy-criterion-sequence-finite-cutoff}
  for all finite $J\subset I$
  \begin{equation*}
    \begin{split}
    \big\| (v_i)_{i\in I} \big\|_J & \leq
    \big\| (v_i^N)_{i\in I} \big\|_J + \big\| (v_i)_{i\in I} - (v_i^N)_{i\in I} \big\|_J = \\
    & = \big\| (v_i^N)_{i\in I} \big\|_J + \lim_{m\to \infty} \big\| (v_i^m)_{i\in I} - (v_i^N)_{i\in I} \big\|_J
    \leq \big\| (v_i^N)_{i\in I} \big\| + 1.
    \end{split}
  \end{equation*}
    
  Hence the family $\left(\big\| (v_i)_{i\in I} \big\|_J\right)_{J\in \mathscr{F}(I)}$
  is bounded and $(v_i)_{i\in I}$ lies in the Hilbert direct sum of the spaces $\hilbertH_i$, $i\in I$. 
  Moreover, \eqref{eq:cauchy-criterion-sequence-finite-cutoff} entails that
  \[
    \big\| (v_i)_{i\in I} - (v_i^n)_{i\in I} \big\|_J =
    \lim_{m\to \infty} \big\| (v_i^m)_{i\in I} - (v_i^n)_{i\in I} \big\|_J \leq \varepsilon \quad
    \text{for all } J\in \mathscr{F}(I) \text{ and }  n\geq N_\varepsilon \ . 
  \] 
  Since $\big\| (v_i)_{i\in I} - (v_i^n)_{i\in I} \big\|$ is the limit of the
  net $\left(\big\| (v_i)_{i\in I} - (v_i^n)_{i\in I} \big\|_J\right)_{J\in\mathscr{F}(I)}$,
  the estimate 
  \[
    \big\| (v_i)_{i\in I} - (v_i^n)_{i\in I} \big\| \leq \varepsilon \quad
    \text{for all }  n\geq N_\varepsilon 
  \]
  follows and the sequence $\left((v_i^n)_{i\in I} \right)_{n\in \N}$ convergence to
  $(v_i)_{i\in I}$. This finishes the proof.    
\end{proof}