% Copyright 2018 Markus J. Pflaum and Daniel Spiegel, licensed under GNU FDL v1.3
% lectures notes taken by Daniel Spiegel based on a lecture by Markus J.Pflaum
% 
\section{Spectral theory of bounded operators}
\para
We now apply the foundations of Hilbert space theory built in the previous sections to 
spectral theory. For the moment we will sacrifice generality and work only with bounded
linear operators. The spectral theory of unbounded linear operators will be treated later. 

Let us a recall that a linear map $A: \hilbertH_1 \rightarrow \hilbertH_2$ between Hilbert spaces 
is continuous if and only if it is bounded, i.e.\ has finite operator norm, and that 
$\blinOps (\hilbertH_1,\hilbertH_2)$ is a Banach space with the operator norm. 
For the rest of this section, $\hilbertH$, $\hilbertH_1$, $\hilbertH_2, \ldots$ will always 
denote complex Hilbert spaces and $A$, $B$ bounded linear operators. 
We will also now fix the base field to be complex, i.e.\ $\fldK = \C$.
Last we agree on writing $I_\hilbertH$ or just $I$ for the identity operator on a Hilbert space 
$\hilbertH$. 

\subsec{Spectrum and Resolvent}

\begin{definition}
  Let $A:\hilbertH \to \hilbertH$ be a bounded linear operator.  A complex number $\lambda$
  is then called an \emph{eigenvalue} of $A$ if there exists a nonzero $v \in H$ such that
  $Av = \lambda v$. For every $\lambda \in \C$ one defines the $\lambda$-\emph{eigenspace} of
  $A$ as
  \[
   \operatorname{Eig}_\lambda  (A) = \big\{ v \in H \bigmid Av = \lambda v \big\} \subset \hilbertH,
  \]
 which is clearly a linear subspace of $\hilbertH$. 
\end{definition}

\para
  By definition it is immediately clear that 
 \[
   \operatorname{Eig}_\lambda (A) = \ker(A - \lambda),
 \]
 where the $\lambda$ on the right stands for the operator $\lambda I$. 
 In other words this means that
 $\lambda\in \C$ is an eigenvalue of $A$ if and only if $A - \lambda$ is not injective.

\begin{definition}
Let $A \in \blinOps(\hilbertH)$. We make the following definitions.
\begin{romanlist}
  \item A \emph{regular value} of $A$ is a complex number $\lambda$ such that $A-\lambda$ is invertible.
  \item The set of all regular values is the \emph{resolvent} of $A$, denoted $\rset(A)$.
  \item A \emph{spectral value} of $A$ is a complex number $\lambda$ such that $A - \lambda$ is not 
        invertible.
  \item The set of all spectral values is the \emph{spectrum} of $A$, denoted $\sigma(A)$.
  \item The \emph{point} or \emph{eigenspectrum} of $A$ is the set
	\[
	\sigma_\textup{p}(A) = \big\{\lambda \in \C \bigmid \ker(A - \lambda) \neq \{0 \} \big\}.
	\]
  \item An \emph{approximate eigenvalue} of $A$ is a complex number $\lambda$ for which there exists 
        a sequence of unit vectors $(v_n)_{n\in \N}\subset \hilbertH$ such that
	\[
   	   \lim_{n \rightarrow \infty} (A - \lambda)v_n = 0.
	\]
	The set $\sigma_\textup{ap}(A)$ is the set of all approximate eigenvalues.
\end{romanlist}
\para 
Evidently, $\sigma(A) = \C \setminus \rset(A)$ and $\sigma_\textup{p}(A) \subset \sigma_\textup{ap}(A) \subset \sigma(A)$,
and these may all be strict inclusions. Note that $A - \lambda$ is bounded for any $\lambda \in \C$, 
so the open mapping theorem \ref{} implies that $(A - \lambda)^{-1} \in \blinOps(\hilbertH)$ when $\lambda \in \rset(A)$.
We call the map 
\[
  R_\bullet (A ):\rset(A) \rightarrow \blinOps(\hilbertH), \quad R_\lambda (A) = (A-\lambda)^{-1}
\]
the \emph{resolvent} of $A$, not to be confused with the resolvent set $\rset(A)$. 
To keep the notation clean, we often briefly write $R_\lambda$ for $R_\lambda(A)$ and leave implicit that $R_\lambda$ depends on $A$.
\end{definition}
First, we prove some topological properties of the spectrum and resolvent. Recall the following lemma, which generalizes the geometric series.

\begin{lemma}[Carl Neumann]\label{thm:neumann-series} Let $A \in \blinOps(\hilbertH)$. If $\norm{A} < 1$, then $I - A$ is invertible,
\[
(I - A)^{-1} = \sum_{n=0}^\infty A^n,
\]
and
\[
\norm{(I-A)^{-1}} \leq \frac{1}{1 - \norm{A}}.
\]
\end{lemma}

\begin{proof}
  Since $\norm{A} < 1$ and $\norm{A^n} \leq \norm{A}^n$ by submultiplicativity of the operator norm, we know
  $\sum_{n=0}^\infty \norm{A^n} < \infty$. This implies that the  family $(A^n)_{n\in\N}$ is absolutely summable,
  so $\sum_{n=0}^\infty A^n$ exists. Furthermore, for every $N \in \N$ we have
  \begin{align*}
    (I-A)\sum_{n=0}^N A^n = \left(\sum_{n=0}^N A^n \right)(I-A)= \sum_{n=0}^N A^n - \sum_{n=1}^{N+1}A^n = I - A^{N+1},
  \end{align*}
which implies that 
\[
\lim_{N \rightarrow \infty}(I-A)\sum_{n=0}^N A^n = \lim_{N \rightarrow \infty} \left(\sum_{n=0}^N A^n\right) (I-A) = I.
\]
By continuity of multiplication in $\blinOps(\hilbertH)$ one gets
\[
(I-A) \sum_{n=0}^\infty A^n = \left(\sum_{n=0}^\infty A^n\right) (I-A) = I,
\]
which proves that $I-A$ is invertible and $(I-A)^{-1} = \sum_{n=0}^\infty A^n$.

Finally,  one concludes by the triangle inequality and submultiplicativity of the operator norm 
\[
  \norm{(I - A)^{-1}} \leq \sum_{n=0}^\infty \norm{A^n} \leq \sum_{n=0}^\infty \norm{A}^n = \frac{1}{1 - \norm{A}}. 
\]
\end{proof}

\begin{proposition}\label{thm:resolvent-topological-properties}
Let $A \in \blinOps(\hilbertH)$. 
\begin{romanlist}
\item For any $\lambda \in \rset(A)$, one has
	\[
	B_{\norm{R_{\lambda}}^{-1}}(\lambda) \subset \rset(A) \ .
	\]
	Hence, $\rset(A) \subset \C$ is open.
\item The spectrum $\sigma(A)$ is compact and
	\[
	\sigma(A) \subset \closure{B}_{\norm{A}}(0) \ .
	\]
\item\label{ite:resolvent-expansion-large-argument} If the complex number $\lambda$ satisfies $\abs{\lambda} > \norm{A}$, then $\lambda \in \rset(A)$ and
      \[
	R_\lambda = - \frac{1}{\lambda} - \sum_{n=1}^\infty \lambda^{-n-1}A^n \ ,
      \]
      where convergence is with respect to the operator norm.  
\end{romanlist}
\end{proposition}

\begin{proof}
 \begin{adromanlist}
 \item 
 Fix $\lambda \in \rset(A)$ and set $r = \norm{R_{\lambda}}^{-1}$. Let $\mu \in B_r(\lambda)$. Then
\begin{align*}
 \norm{(\mu - \lambda)R_{\lambda}} = \abs{\mu - \lambda} \norm{R_{\lambda}} < 1.
\end{align*}
Thus, by Lemma \ref{thm:neumann-series}, one knows that $I - (\mu - \lambda)R_{\lambda}$ is invertible. Since $A - \lambda$ is invertible,
the composition
\[
(A - \lambda) \, \big( I - (\mu - \lambda)R_{\lambda} \big) = A - \mu
\]
is invertible, which proves that $\mu \in \rset(A)$. Hence $\rset(A)$ is open.


\item Since $\rset(A)$ is open, the complement $\sigma(A) = \C \setminus \rset(A)$ is closed. Furthermore, if $\abs{\lambda} > \norm{A}$, then $\norm{\lambda^{-1}A} < 1$, so $I - \lambda^{-1}A$ and hence $A - \lambda$ are invertible by Lemma \ref{thm:neumann-series}. This implies that $\lambda \in \rset(A)$, so $\sigma(A) \subset \closure{B}_{\norm{A}}(0)$. Since $\sigma(A)$ is closed and bounded, it is compact.
\item If $\abs{\lambda} > \norm{A}$, then $I - \lambda^{-1}A$ is invertible by Lemma \ref{thm:neumann-series} and
\[
(I - \lambda^{-1}A)^{-1} = \sum_{n=0}^\infty \lambda^{-n}A^n.
\]
Since $-\lambda(A - \lambda)^{-1} = (I - \lambda^{-1}A)^{-1}$, one obtains
\[
R_\lambda = -\frac{1}{\lambda}\sum_{n=0}^\infty \lambda^{-n}A^n = -\frac{1}{\lambda} - \sum_{n=1}^\infty \lambda^{-n-1}A^n,
\]
as desired.   
\end{adromanlist}
\end{proof}

Next, we prove some algebraic properties of the resolvent. Hereby, $[A,B] = AB - BA$ denotes the commutator of two operators,
as usual.

\begin{proposition}\label{thm:resolvent-algebraic-properties}
Let $A,B \in \blinOps(\hilbertH)$. Then the following holds true.
\begin{romanlist}
\item\label{ite:commutativity-resolvent-operator} The resolvent commutes with the operator which means that
  \[ [A, R_\lambda(A) ] = 0 \quad\text{for all } \lambda \in \rset(A) \ . \]
\item\label{ite:commutativity-resolvent-itself}
   The values of the resolvent commute with each other that is
 \[ [R_\lambda (A), R_\mu (A)] = 0 \quad\text{for all } \lambda,\mu \in \rset(A) \ . \]
\item\label{ite:first-resolvent-identity}\textup{({\sffamily First resolvent identity})} For all $\lambda, \mu \in \rset(A)$
  \[ R_\lambda(A) - R_\mu (A)= (\lambda - \mu)R_\lambda(A) R_\mu(A) \ . \]
\item\textup{({\sffamily Second resolvent identity})} For all $\lambda \in \rset(A)\cap \rset(B)$
  \[ R_\lambda (A)- R_\lambda (B) = R_\lambda(A)\, (B-A) \, R_\lambda (B) \ . \] 
\end{romanlist}
\end{proposition}

\begin{proof}
\begin{adromanlist}
\item Obviously  $[A, A - \lambda] = 0$, so
  \[
    0 = R_\lambda [A, A - \lambda]R_\lambda = R_\lambda A - AR_\lambda,
  \]
as desired.
\setcounter{enumi}{2}
\item We compute
\begin{align*}
(R_\lambda - R_\mu)(A - \mu)(A - \lambda) &= (R_\lambda A - \mu R_\lambda)(A - \lambda) - (A - \lambda)\\
&= (A - \mu)R_\lambda (A - \lambda) - (A - \lambda)\\
&= \lambda - \mu ,
\end{align*}
where we used part \ref{ite:commutativity-resolvent-operator} to commute $R_\lambda$ past $A$ in the second step.
Now multiplying both sides with $R_\lambda R_\mu $ from the right yields the desired equality. 
\setcounter{enumi}{1}
\item 
  For $\lambda = \mu$, one obviously has $[A_\lambda, A_\mu] = 0$. For $\lambda \neq \mu$, one concludes from
  \ref{ite:commutativity-resolvent-itself}
  \[
     R_\mu R_\lambda = \frac{R_\mu - R_\lambda}{\mu - \lambda} = \frac{R_\lambda - R_\mu}{\lambda - \mu} = R_\lambda R_\mu,
  \]
  so $[R_\lambda, R_\mu] = 0$ for $\lambda \neq \mu $ as well.
\setcounter{enumi}{3}
\item  The  last equality follows by
  \[ R_\lambda(A)\, (B-A) \, R_\lambda (B) =  R_\lambda(A)\, \big((B-\lambda)-(A-\lambda) \big) \, R_\lambda (B) =
     R_\lambda(A) - R_\lambda (B) \ . \]
\end{adromanlist}
\end{proof}

The resolvent $R_\bullet (A)$ also has some nice analytic properties which we are going to prove next. 
\begin{proposition}\label{thm:resolvent-analytic-properties}
  The resolvent $R_\bullet (A) :\rset(A) \rightarrow \blinOps(\hilbertH)$, $\lambda \mapsto R_\lambda$ is continuous and complex
  differentiable with derivative given by 
  \[
    R_\bullet(A)' :\: \rset(A) \rightarrow \blinOps(\hilbertH),  \: \lambda \mapsto \lim_{\mu \rightarrow \lambda} \frac{R_\mu - R_\lambda}{\mu - \lambda} =  R_\lambda^2
  \]
\end{proposition}

\begin{proof}
Fix $\lambda \in \rset(A)$ and $\varepsilon > 0$. Let $0 < \abs{\mu - \lambda} < \delta$, where 
\[
  \delta = \min\left( \frac{\varepsilon}{2\norm{R_\lambda}^2},\, \frac{1}{2\norm{R_\lambda}} \right) \ .
\]
Note that $\mu \in \rset(A)$ by Proposition \ref{thm:resolvent-topological-properties}. Moreover, $\norm{(\mu - \lambda)R_\lambda} <1$, so $I - (\mu - \lambda)R_\lambda$ is invertible with norm less than $(1 - \norm{(\mu - \lambda) R_\lambda})^{-1}$ by Lemma \ref{thm:neumann-series}.
Now observe that the first resolvent identity can be rearranged to
\[
R_\mu =  R_\lambda[I - (\mu - \lambda)R_\lambda]^{-1} \ .
\]
Hence
\begin{align*}
\norm{R_\mu - R_\lambda} &\leq \abs{\mu - \lambda}\norm{R_\mu}\norm{R_\lambda} \\
&\leq \abs{\mu - \lambda} \norm{R_\lambda}^2 \norm{(I - (\mu - \lambda)R_\lambda)^{-1}}\\
&\leq \frac{\abs{\mu - \lambda} \norm{R_\lambda}^2}{1 - \norm{(\mu - \lambda)R_\lambda}}\\
&< \frac{\varepsilon/2}{1-1/2} = \varepsilon \ .
\end{align*}
This proves that $\lambda \mapsto R_\lambda$ is continuous. 

As for complex differentiability, we simply use the first resolvent identity and continuity to conclude
\[
\lim_{\mu \rightarrow \lambda} \frac{R_\mu - R_\lambda}{\mu - \lambda} = \lim_{\mu \rightarrow \lambda} R_\mu R_\lambda = R_\lambda^2. 
\]
\end{proof}

\begin{proposition}\label{thm:resolvent-limes-infinity}
Let $A \in \blinOps(\hilbertH)$. Then $\lambda R_\lambda \rightarrow -I$ as $\abs{\lambda} \rightarrow \infty$. In particular, $R_\lambda \rightarrow 0$ as $\abs{\lambda} \rightarrow \infty$.
\end{proposition}

\begin{proof}
Fix $\varepsilon > 0$.  For $\abs{\lambda} > \norm{A}$, we have by \Cref{thm:resolvent-topological-properties} \ref{ite:resolvent-expansion-large-argument}
\[
\lambda R_\lambda = -I - \sum_{n=1}^\infty \lambda^{-n}A^n.
\]
Since 
\[
\norm{\sum_{n=1}^\infty \lambda^{-n}A^n} \leq  \frac{\norm{A}}{\abs{\lambda}-\norm{A}},
\]
one sees that $\lambda R_\lambda \rightarrow - I$ as $\abs{\lambda} \rightarrow \infty$. Similarly, for $\abs{\lambda} > \norm{A}$ one has
\[
\norm{R_\lambda} \leq \frac{1}{\abs{\lambda}} + \frac{1}{\abs{\lambda}}\sum_{n=1}^\infty \norm{\lambda^{-n}A^n} \leq \frac{1}{\abs{\lambda}} + \frac{1}{\abs{\lambda}} \frac{\norm{A}}{\abs{\lambda} - \norm{A}},
\]
which shows that $R_\lambda \rightarrow 0$ as $\abs{\lambda} \rightarrow \infty$.
\end{proof}

\begin{proposition}
For all $v, w \in \hilbertH$, the map 
\[
 \inprod{ R_\bullet (A) v,  w} : \: \rset(A) \rightarrow \C , \: \lambda \mapsto \inprod{ R_\lambda v, w}
\]
is holomorphic with derivative
\[
 \inprod{ R_\bullet (A) v, w}' : \: \rset(A) \rightarrow \C , \: \lambda \mapsto \inprod{R_\lambda^2 v,  w}.
\]
\end{proposition}

\begin{proof}
Given $\lambda \in \rset(A)$, we compute
\begin{align*}
  \lim_{\mu \rightarrow \lambda} \frac{\inprod{R_\mu v, w} - \inprod{R_\lambda v,  w}}{\mu - \lambda} =
  \lim_{\mu \rightarrow \lambda} \frac{\inprod{(\mu - \lambda)R_\mu R_\lambda v,  w}}{\mu - \lambda} =
  \lim_{\mu \rightarrow \lambda} \inprod{ R_\mu R_\lambda v, w} = \inprod{R_\lambda^2v , w},
\end{align*}
where we have used the first resolvent identity in the first step and continuity of the inner product in the last.
\end{proof}

\begin{proposition}
The spectrum of an operator $A \in \blinOps(\hilbertH)$ is nonempty.
\end{proposition}

\begin{proof}
Suppose $\sigma(A) = \emptyset$, hence $\rset(A) = \C$. The map
\[
\C \to \C , \: \lambda \mapsto \inprod{R_\lambda v, w}
\]
then is entire for every $v, w \in \hilbertH$. Furthermore, one has  for $\norm{v}, \norm{w} \leq 1$
\[
  \abs{\inprod{R_\lambda v,  w}} \leq  \norm{R_\lambda} \norm{v} \norm{w} \leq \norm{R_\lambda} \ .
\]
Since $\lambda \mapsto \norm{R_\lambda}$ is continuous and $\norm{R_\lambda} \rightarrow 0$ as $\abs{\lambda} \rightarrow \infty$, one sees
that $\norm{R_\lambda}$ is bounded. Hence $\inprod{R_\bullet v,  w}$ is a bounded entire function, which  by Liouville's theorem implies
that it is zero for every pair $v, w \in \hilbertH$ with $\norm{v} = \norm{w} = 1$.
This entails that $R_\lambda = 0$ for every $\lambda \in \C$, which 
is a contradiction to $R_\lambda$ being invertible. Hence $\sigma(A) \neq \emptyset$.
\end{proof}


