% Copyright 2018 Markus J. Pflaum, licensed under CC BY-NC-ND 4.0
% main author: 
%   Markus J. Pflaum
%
\section{The category of topological vector spaces}
\label{sec:vector-space-topologies-local-convexity}

\subsec{Vector space topologies}
%\para
% Unless mentioned differently, $R$ will denote in this section a division
% ring with a non-trivial absolute value $|\cdot|:R\to\R$.
% By a vector space over $R$ we always mean a left $R$-module. 
\begin{definition}
  \label{def:topological-vector-space}
   Let $R$ be a topological division ring. 
   A topology $\topology$ on  a vector space  $\tvsE$  over $R$ 
   is called a \emph{vector space topology} if the following axioms hold true:
   \begin{axiomlist}[TVS]
   \item\label{axiom:tvs-continuity-addition}
     Addition $+ : \tvsE \times \tvsE \to \tvsE$ is continuous.
   \item\label{axiom:tvs-continuity-multiplication-scalars} 
     Multiplication by scalars  $\cdot : R \times \tvsE \to \tvsE$ is 
     continuous.
   \end{axiomlist}
   The topology $\topology$ on $\tvsE$ is called \emph{translation invariant}
   if for every $w\in \tvsE$ the linear map 
   $\ell_{w} : \vectorspE \to \vectorspE$, $v\mapsto v +w$ is a homeomorphism.

   A vector space  $\tvsE$ endowed with a vector space topology on it
   is called a \emph{topological vector space} (\emph{over} $R$), for short a \tvs
\end{definition}

\begin{remark}
  Let us recall at this point some notation from linear algebra. Assume that $\vectorspV$ is
  a left vector space over the divison ring $R$.
  If $A,B\subset \vectorspV$ are two non-empty subsets, then $A+B$ is
  the set of all $v\in \vectorspV$ for which there exist $x\in A$ and $y\in B$ such that
  $v = x+y$. If $A$ or $B$ is empty, then $A+B$ is defined as the empty set.
  In case $A$ is a singleton that is if $A =\{x\}$, then we often write $x + B$ instead of
  $\{x\} +B$. If $\mathscr{B} \subset \power{\vectorspV}$ is a non-empty set of subsets of $\vectorspV$,
  then we denote by $A + \mathscr{B}$ and $x + \mathscr{B}$ the sets
  $\{ A + B \in \power{\vectorspV} \mid  B \in  \mathscr{B}\} $ and
  $\{ x + B \in \power{\vectorspV} \mid  B \in  \mathscr{B}\} $, respectively.
  If $\mathscr{A} \subset \power{\vectorspV}$  is a  second non-empty set of subsets of $\vectorspV$,
  then $\mathscr{A} + \mathscr{B}$ stands for the set of all sets of the form $A+B$, where
  $A\in \mathscr{A}$ and $B\in \mathscr{B}$. 
  
  In case $C$ is a subset of the ground ring $R$, then $C \cdot A$ is defined as the set of all
  $v\in \vectorspV$ for which there exist
  $r\in C$ and $x\in A$ such that $v =r \cdot x$. If $r\in R$ we write $r\cdot A$
  for $\{r\} \cdot A$. Likewise, if $x \in \vectorspV$, $C\cdot x$ stands for $C\cdot\{x\}$.
  Analogously as for addition the sets $\mathscr{C}\cdot A$, $C\cdot \mathscr{A}$
  and $\mathscr{C}\cdot \mathscr{A}$ are defined when  
  $\mathscr{C} \subset \powerset{R}$ and $\mathscr{A} \subset \power{\vectorspV}$
  are non-empty. 
\end{remark}

\begin{proposition}\label{thm:basic-properties-sets-maps-topological-vector-spaces}
  Let $\vectorspE$ be a \tvs~over a topological division ring $R$.
  Then the following holds true:
  \begin{romanlist}
  \item\label{ite:translated-homotheties-homeomorphisms}
    For every $r\in R^\times$ and $w\in \vectorspE$ the homothety
    $\ell_{r,w} : \vectorspE \to \vectorspE$, $v\mapsto rv +w$ is a homeomorphism
    with inverse $\ell_{r^{-1},-r^{-1}w}$. 
   \item\label{ite:translated-base}
     Let $w$ be an element of $\vectorspE$ and $r\in R^\times$. 
     A filter base $\base$ on $\vectorspE$ then is a filter base for the zero
     neighborhoods if and only if  $w + r \base$ is a filter base for the
     neighborhoods of $w$.
   \item\label{ite:closure-terms-base}
     If $\base$ is a filter base of the filter of zero neighborhoods, then 
     the closure of any non-empty $A\subset \vectorspE$  is given by
     \[
       \closure{A} =\bigcap\limits_{U \in \base} A + U \ .
     \]
   \item\label{ite:sum-open}
     Let $A\subset \vectorspE$ be open and $B\subset \vectorspE$. Then the set
     $A+B$ is open.
   \item\label{ite:sum-closed}
     Let $A, B \subset \vectorspE$ be closed and assume that $A$ is quasi-compact
     that is that any filter on $A$ has a cluster point.
     Then the set $A+B$ is closed.
   \item
     The space  $\vectorspE$ is \ref{axiom:t3} or, equivalently, each point of $\vectorspE$ possesses a
     neighborhood base consisting of closed subsets.
  \end{romanlist}   
  \end{proposition}

\begin{proof}
  \begin{adromanlist}
  \item
    The homothety $\ell_{r,w}$ is continuous since addition and multiplication by a scalar
    are continuous maps on a \tvs\ Since for all $v\in V$ 
    \begin{align}
      \nonumber
      & \ell_{r^{-1},-r^{-1}w}\circ \ell_{r,w} (v) = r^{-1} (rv +w) - r^{-1}w = v, \text{ and}\\
      \nonumber
      & \ell_{r,w} \circ \ell_{r^{-1},-r^{-1}w} (v) = r (r^{-1}v  - r^{-1}w) + w = v
    \end{align}
    the homothety $\ell_{r,w}$  is invertible, and its inverse is $\ell_{r^{-1},-r^{-1}w}$.
  \item
    This follows since $\ell_{r,w}$ is a homeomorphism.   
  \item
     Let $B = \bigcap\limits_{U \in \base} A + U$. Let $v$ be an element of the closure of $A$.
     Then, for $U \in \base$, there exists an element $a \in A \cap v - U$ by \ref{ite:translated-base}
     and since $-U$ is a zero neighborhood. Hence $v \in a + U$, and $\closure{A} \subset B$ follows.
     Now let $v \in B$ and $V$ be a neighborhood of $v$. Then there exists $U\in \base$ such that 
     $v -U \subset V$. By definition of $B$ there exists an element $a\in A$ such that $v \in a + U$.
     Hence $a \in v - U \subset V$ which implies that $v \in \closure{A}$.
     So $B \subset \closure{A}$. 
  \item
    The set $A+B$ is either empty or coincides with the union $\bigcup_{v\in B} v + A$.
    In the latter case, each of the sets  $v+ A$ is non-empty and
    open by continuity of addition. So $A+B$ is open under the assumptions made.
  \item
    We can assume that $A$ and $B$ are non-empty because the claim is trivial otherwise.
    Assume that $A+B$ is not closed. Then there exists an element $v \in \vectorspE \setminus (A+B) $
    such that each neighborhood of $v$ meets $A+B$. This means in particular that
    the restriction of the neighborhood filter $\nbhdfilter$ of $v$ to $A+B$ is a filter base.
    Consequently, $(- B + \nbhdfilter)\cap A$ is a filter base on $A$, hence possesses an
    accummulation point $x \in A$. For each neighborhood $V \in \nbhdfilter$ the point $x$ is then
    contained in the closure of $-B +V$. Hence, by \ref{ite:closure-terms-base},
    $x$ is contained in $v -B + U + U$ for every zero neighborhood $U$.
    Since by continuity of addition $U + U$ runs through a base of zero neighborhoods when
    $U$ runs through the zero neighborhoods,
    $x \in v - \closure{B} = v -B$ follows. Since $x \in A$ this contradicts the assumption
    $v \in A+B$ and $A+B$ has to be closed.
  \item
    Let $v \in \vectorspE$, $A\subset \vectorspE$ closed, and assume $v \notin A$.
    Choose an open neighborhood $V$ of $v$ such that $V \cap A = \emptyset$. Then there
    exists an open zero neighborhood $U$ such that $v + U + U \subset V$. By possibly passing to
    $U \cap (-U)$ we can assume that $U = -U$. Now $v+U$ is an open neighborhood of $v$
    and $A+U$ one of $A$. These neighborhoods are disjoint because if the intersection
    $v+U \cap A+U$ is non-empty, then there exists an element $w \in v+U + U\cap A$ since
    $-U =U$. This contradicts  $V \cap A = \emptyset$, so   $v+U$ and $A+U$ are disjoint neighborhoods
    of $v$ and $A$, respectively. Hence $\vectorspE$ satisfies \ref{axiom:t3}.
  \end{adromanlist}
\end{proof}

\begin{corollary}
  Every vector space topology on a vector space over a topological division ring
  is translation invariant.
\end{corollary}

\begin{proof}
  This follows immediately by
  \Cref{thm:basic-properties-sets-maps-topological-vector-spaces}
  \ref{ite:translated-homotheties-homeomorphisms}.
\end{proof}

\begin{definition}
  A subset $C$ of a  vector space  $\vectorspE$  over a valued division ring $(R,|\cdot|)$ is called
  \begin{romanlist}
  \item
    \emph{symmetric} if $-v \in C$ for all $v\in C$,
  \item
    \emph{circled} or \emph{balanced} if $rv \in C$ for all $v\in C$ and $r\in R$ with $|r| \leq 1$.
  \end{romanlist}
\end{definition}

\begin{remark}
  Symmetry of a subset of a vector space of a division ring is even defined when the underlying division
  ring does not carry an absolute value. 
\end{remark}

\begin{lemma}
\label{thm:closure-interior-circled-set-circled}
  Let $C$ be a subset of a topological vector space $\vectorspE$ over a
  valued division ring $(R,|\cdot|)$ and $r\in R$.
  \begin{romanlist}
  \item\label{ite:interior-closure-symmetric-sets-symmetric}  
    If $C$ is symmetric, then the closure $\closure{C}$ and the interior $\interior{C}$ are symmetric.  
  \item\label{ite:interior-closure-circled-sets-circled}    
    If $C$ is circled, then the closure $\closure{C}$ and  the union $\interior{C} \cup\{0\}$ are circled.
  \item\label{ite:stretching-preserving-symmetric-circled-sets}  
    The set $rC$ is symmetric (respectively circled) if $C$ has that property.  
  \end{romanlist}
\end{lemma}

\begin{proof}
  Without loss of generality we can assume $C \neq \emptyset$.
  Claim \ref{ite:interior-closure-symmetric-sets-symmetric} then follows immediately since multiplication
  by $-1$ is a homeomorphism. 
  To prove  claim \ref{ite:interior-closure-circled-sets-circled} assume that $C$ is circled. 
  Let $s\in R$ with $|s|\leq 1$. Assume $v \in \closure{C}$ and consider $sv$. We have to show
  that $sv \in \closure{C}$. If $s=0$ then $sv =0\in C \subset \closure{C}$ since $C$ is circled.
  So we can assume $s\neq 0$ and need to show that for every neighborhood $V$ of $sv$ the intersection
  $C\cap V$ is non-empty. Since $|s|>0$, the homothety $\ell_s:\vectorspE\to \vectorspE$, $w\mapsto sw$ is a
  homeomorphism with inverse
  $\ell_{s^{-1}}$. Hence $s^{-1}V$ is a neighborhood of $v$. Since $v$ lies in the closure of $C$ 
  there exists an element $w\in C\cap s^{-1}V$. Hence $sw \in C \cap V$ by assumption on $C$
  and $\closure{C}$ is circled.

  If $v \in \interior{C}\cup\{0\}$ then $0 = 0 \cdot v \in \interior{C}\cup\{0\}$. It remains to show that
  $sv \in \interior{C}\cup\{0\}$ for $s\in R$ with $0 < |s|\leq 1$ and $v \in \interior{C} \setminus \{ 0\}$.
  Under this assumption the homothety $\ell_s$ is a homeomorphism, so $s \interior{C}$ is an open subset
  of $C$ since $C$ is circled. Hence $s v \in s \interior{C} \subset \interior{C}$, and  $\interior{C}\cup\{0\}$
  is circled as well.

  Claim \ref{ite:stretching-preserving-symmetric-circled-sets} follows immediately from the observation that
  for  $v\in C$ and $s\in R$ the relation $srv \in rC$ holds true if $sv \in C$. 
\end{proof}

\begin{propanddef}\label{thm:symmetric-circled-hulls}
  The intersection of a non-empty family $(C_i)_{i\in I}$ of symmetric (respectively circled) subsets $C_i \subset \vectorspE$, $i\in I$
  of a topological vector space $\vectorspE$ over a valued division ring $(R,|\cdot|)$
  is symmetric (respectively circled).
  In particular, if $A \subset \vectorspE$ is a subset, then the sets
  \[
    \symHull A = \bigcap\limits_{A \subset B \subset \vectorspE \atop {B \text{ is symmetric}} }  B \quad \text{and} \quad
    \circHull A = \bigcap\limits_{A \subset B \subset \vectorspE \atop {B \text{ is circled}} }  B 
  \]
  are symmetric and circled, respectively.   They have the property that
  $\symHull A$ is the smallest symmetric and $\circHull A$ the smallest circled subsets of $\vectorspE$ containing $A$.
  They are called the \emph{symmetric} and the \emph{circled hull} of $A$, respectively.
  Analogously,
  \[
    \clsymHull A = \bigcap\limits_{A \subset B =\closure{B} \subset \vectorspE \atop {B \text{ is symmetric}} }  B \quad \text{and} \quad
    \clcircHull A = \bigcap\limits_{A \subset B  =\closure{B}  \subset \vectorspE \atop {B \text{ is circled}} }  B 
  \]
  are called the \emph{closed symmetric} and the \emph{closed circled hull} of $A$, respectively.
  They have the property that $\clsymHull A$ is the smallest closed symmetric and $\clcircHull A$ the smallest closed
  circled subset of $\vectorspE$ containing $A$.
\end{propanddef}
\begin{proof}
  Note first that all the hulls in the proposition are well-defined since $\vectorspE$ is
  closed and circled. 
  Let $C$ denote the intersection of the family $(C_i)_{i\in I}$. Assume that for some $r\in R$ with $|r|\leq 1$
  the inclusion $rC_i \subset C$ holds true for all  $i\in I$. Then $rC \subset C$, hence if all
  $C_i$ are symmetric (respectively circled), so is $C$.
  This observation now entails that $\symHull A$ is symmetric, $\circHull$ is circled, $\clsymHull A$ is closed
  and symmetric, and finally that $\clcircHull A$ is closed and circled. Moreover, all those sets contain $A$.
  The minimality properties of these sets are clear by construction.  
\end{proof}

\begin{remark}
  Observe that by the proposition $A$ is symmetric if and only if $\symHull A = A$ and
  circled if and only if $\circHull A = A$. Analogously,
  $\clsymHull A = A$ if and only if $A$ is closed symmetric and
  $\clcircHull A = A$ if and only if $A$ is closed and circled. 
\end{remark}


\begin{lemma}
  Let $\vectorspE$ be a topological vector space over the valued division ring $(R,|\cdot|)$
  and $A \subset \vectorspE$ non-empty. Then
  \[
    \symHull A = A \cup -A  \quad \text{and} \quad \circHull A = \bigcup_{r\in R, \: |r|\leq 1} r A \ .
  \]
  For the closed hulls one has  
  \[
    \clsymHull A = \overline{\symHull A} \quad \text{and} \quad \clcircHull A = \overline{\circHull A} \ .
  \]
\end{lemma}

\begin{proof}
  Since $A \cup -A$ is symmetric by definition, contains $A$, and is contained in
  $\symHull A$, the equality $\symHull A = A \cup -A$ holds true.
  Similarly, $\bigcup_{r\in R, \: |r|\leq 1} r A$ is circled by definition, contains $A$, and is contained
  in $\circHull A$ by definition of the circled hull. Hence $ \circHull A = \bigcup_{r\in R, \: |r|\leq 1} r A $.
  The remainder of the claim follows from \Cref{thm:closure-interior-circled-set-circled}.
\end{proof}

\begin{definition}
  Assume that $B,C$ are subsets of a vector space $\vectorspE$ over the valued division ring
  $(R,|\cdot|)$. Then one says that 
  \begin{romanlist}
  \item
   $C$  \emph{absorbes} $B$ if  there exists a real number $t \in \R_{\geq 0}$ 
   such that $B \subset rC$ for all $r\in R$ with $|r| \geq t$,
  \item
   $C$ is  \emph{absorbing} or \emph{absorbent} 
   if $C$ absorbes every one-point set of $\tvsE$ that is if for every 
   $v\in \vectorspE$ there exists $t \in \R_{\geq 0}$ such that 
   $v \in rC$ for all $r\in R$ with $|r| \geq t$.
 \end{romanlist}
 
  If the vector space $\vectorspE$ carries in addition a vector space topology, then one says that
  \begin{romanlist}
  \setcounter{enumi}{2}  
  \item
    the subset $B \subset \vectorspE$ is \emph{bounded} if it is absorbed by
    every zero neighborhood. 
  \end{romanlist}
\end{definition}

\begin{lemma}\label{thm:intersection-scalar-multiples-absorbing-sets-absorbing}
  Let $\vectorspE$ be a vector space over the valued division
  ring $(R,|\cdot|)$. Then the following holds true.
  \begin{romanlist}
  \item\label{ite:intersection-absorbing-sets-absorbing}
    If $C_1,\ldots,C_n$ are absorbing subset of $\vectorspE$, then the intersection
    $C_1\cap\ldots \cap C_n$ is absorbing.
  \item\label{ite:scalar-multiple-absorbing-set-absorbing}
    If $C$ is an absorbing subset of $\vectorspE$, then $rC$ is absorbing for every
    $r\in R^\times$.
  \end{romanlist} 
\end{lemma}

\begin{proof}
  \begin{adromanlist}
  \item
    Let $v\in\vectorspE$ and choose $t_1,\ldots,t_n\in \R_{\geq 0}$ such that
    $v \in rC_i$ for $|r|\geq t_i$. Put $t = \max \{ t_1,\ldots , t_n\}$.
    Then $v \in r (C_1\cap\ldots \cap C_n)$ for $|r|\geq t$, hence $C_1\cap\ldots \cap C_n$ is absorbing.
  \item  
    Choose $t\in \R_{\geq 0}$ such that  $v \in sC$ for all $s\in R$ with $|s|\geq t$.
    Then one has $|sr| \geq t$ for all $s\in R$ with $|s|\geq \frac{t}{|r|}$, hence $v \in s (rC)$
    for all such $s$. Therefore $rC$ is absorbing. 
  \end{adromanlist}
\end{proof}

\begin{proposition}\label{thm:topological-vector-space-zero-neighborhood-filter-base-circled-absorbing}
  The filter of zero neighborhoods of a topological vector space $\tvsE$ over
  $(R,|\cdot|)$ has a filter base $\base$ with the following properties:
  \begin{romanlist}
  \item\label{ite:tvs-base-element-containing-sum-base-elements}
    For each $V \in \base$ there exists $U\in\base$ such that $U+U \subset V$.
  \item\label{ite:tvs-base-elements-circled-balanced}
    Every element $V\in \base$ is circled and absorbing.
  \item\label{ite:tvs-base-element-containing-shrunk-base-element}
    There exists an element $r\in R^\times$ with $0< |r| < 1$ such that $V\in \base$ implies
    $rV \in \base$. 
  \end{romanlist}
  Conversely, if  $\base$ is a filter base on an $R$-vector space $\vectorspE$ such that 
  \ref{ite:tvs-base-element-containing-sum-base-elements} to \ref{ite:tvs-base-element-containing-shrunk-base-element}
  hold true, then there exists a unique vector space topology on $\vectorspE$  such that $\base$ is a neighborhood base
  at the origin.  In case the ground ring $R$ is archimedean, a filter base on $\vectorspE$ which satisfies
  \ref{ite:tvs-base-element-containing-sum-base-elements} and \ref{ite:tvs-base-elements-circled-balanced}
  already induces a unique vector space topology having $\base$ as a neighborhood base
  at $0$.
  In either of these two cases, the thus constructed topology coincides with the coarsest translation invariant topology for which $\base$ is a set of zero neighborhoods.
\end{proposition}

\begin{proof}
  Assume that $\vectorspE$ is a  \tvs\  Let $\base$ be the set of circled neighborhoods of $0$. We show first that
  $\base$ is a base of the filter $\nbhds_0$ of zero neighborhoods. Let $W \in \nbhds_0$. 
  By Axiom \ref{axiom:tvs-continuity-multiplication-scalars} there exists an $\varepsilon >0$ and an open
  zero neighborhood $U$ such that $sU \subset W$ for all $s\in R$ with $|s|<\varepsilon$.
  Then $V = \bigcup\limits_{s\in R^\times \:\&\: |s|<\varepsilon} sU $ is a zero neighborhood since
  by \Cref{thm:zero-neighborhood-topological-division-ring-infinite} the set of $s\in R^\times$
  with $|s|<\varepsilon$ is non-empty. By construction $V$ is contained in $W$ and circled, so $V \in \base$.
  Hence $\base$ is a filter base of $\nbhds_0$.

  Next recall that  there exists $r\in R^\times$ with $0<|r|<1$ since the absolute value $|\cdot|$ is non-trivial.
  Let $V\in \base$. Then $sV\subset V$ for all $s\in R$ with $|s|\leq 1$ which entails $srV \subset rV$ for all
  such $s$. Hence $rV$ is circled and an element of $\base$ as well. This proves
  \ref{ite:tvs-base-element-containing-shrunk-base-element}.
  Since addition on $\vectorspE$ is continuous, there exist for given $V\in \base$ open neighborhoods
  $U_1,U_2$ of the origin such that $U_1+U_2\subset V$. Choose $U\in \base$ such that $U \subset U_1 \cap U_2$.
  Then $U + U \subset V$ and \ref{ite:tvs-base-element-containing-sum-base-elements} is proved.
  To show that any $V\in\base$ is absorbing let $v\in \vectorspE$. By continuity of scalar multiplication
  there exists $\varepsilon >0$  such that $sv \in V $ for all $s\in R$ with $|s|<\varepsilon$.
  By \Cref{thm:basic-properties-sets-maps-topological-vector-spaces} \ref{ite:translated-homotheties-homeomorphisms}
  this entails $v \in sV $ for all $s\in R$ with $|s|>\varepsilon$ and $V$ is absorbing. 

  Now assume that $\vectorspE$ is an $R$-vector space and that $\base$ is a filter base that satisfies
  \ref{ite:tvs-base-element-containing-sum-base-elements}, \ref{ite:tvs-base-elements-circled-balanced} and,
  if $|\cdot|$ is non-archimedean, \ref{ite:tvs-base-element-containing-shrunk-base-element}.
  Since $\base$ consists of non-empty circled sets, $0\in V$ for all $V\in\base$. Let
  $\topology \subset \powerset{\vectorspE}$ be the set of all $U\subset \vectorspE$ such that
  for each $v\in U$ there exists $V\in \base$ with $v+V \subset U$. By definition and since $\base$ is a
  filter base, $\topology$ is a topology on $\vectorspE$.
  By construction, $\topology$ is also the coarsest translation invariant topology for which $\base$ is a set of zero neighborhoods. We show that $\base$ is a base of the filter
  $\nbhds_0$ of zero neighborhoods. By definition of $\topology$ there exists for each $U \in \nbhds_0$
  a $V\in \base$ such that $V\subset U$. So it remains to show that each $V\in \base$ is a zero neighborhood.
  To this end let $U$ be the set of all $v\in V$ for which there exists
  a $W\in \base$ with $v + W \subset V$. Since $0 +V \subset V$ one has $0\in U$. The relation 
  $U \subset V$ holds because  $0\in W$ for all $W\in \base$.  Now let $v \in U$.
  By \ref{ite:tvs-base-element-containing-sum-base-elements} there exists $W^\prime$ such that 
  $v + W^\prime + W^\prime \subset V$ which entails $v + W^\prime \subset U$. Hence $U\in\topology$
  and $V$ is a zero neighborhood. 
  Next we verify that $\topology$ is a vector space topology. We start with continuity of addition.   
  Let $W$ be an open neighborhood of $v+w$, where $v,w\in \vectorspE$. Then there exists $V\in \base$ such that
  $v+w + V \subset W$. Choose $U\in \base $ such that $U+U \subset V$. Then  
  $v+U$ and $w +U$ are neighborhoods of $v$ and $w$, respectively, and $(v+U) + (w +U) \subset v+w +V\subset W$.
  So addition is continuous. We continue with scalar multiplication. Let $W$ be an open neighborhood of $rv$,
  where $r\in R$ and $v\in \vectorspE$. Then there exists $V \in \base $ such that $rv + V + V \subset W$.
  Since $V$ is absorbing by \ref{ite:tvs-base-elements-circled-balanced} there exists $\varepsilon >0$ such that
  $ (s-r)v \in V$ for all $s\in R$ with  $|s-r| < \varepsilon$. Now if $|\cdot|$ is non-archimedean
  choose $t\in R^\times$ according to \ref{ite:tvs-base-element-containing-shrunk-base-element}, 
  and put $V_n = t^n V$ for all $n\in\N$.
  In the archimedean case let $t = \frac 12$ and use \ref{ite:tvs-base-element-containing-sum-base-elements} to construct
  recursively a sequence $(V_n)_{n\in \N }$ of elements of $\base$ such that
  $2^nV_n = V_n + \ldots + V_n \subset V$, where the sum has $2^n$ summands. In either of these cases, choose 
  $N\in \N$ large enough so that $|t|^N < \frac{1}{|r| + \varepsilon}$. Then $V_N \in \base$
  and $v + V_N$ is a neighborhood of $v$. Moreover, for $w \in v + V_N$ there exists an element
  $x\in V$ such that $w-v = t^N x$. Then the relation
  $s (w-v) = s t^N x \in V $ holds whenever $|s-r| < \varepsilon$ since $V_N$ is circled. Hence for such $w$ and $s$
  \[
     sw = rv + s (w-v) + (s-r)v \in  rv + V + V \subset W \ . 
  \]
  This means that scalar multiplication is continuous, and the proof is finished. 
\end{proof}

\subsec{Morphisms of topological vector spaces}
\begin{definition}
  By a \emph{morphism} of topological vector spaces over the topological division
  ring $R$  one understands  a continuous $R$-linear map $f:\vectorspE \to\vectorspF$
  between two topological vector spaces $\vectorspE$ and $\vectorspF$
  over $R$. The space of morphisms between $\vectorspE$ and $\vectorspF$ will be denoted
  $\Hom_{R\text{-}\category{TVS}} (\vectorspE,\vectorspF)$ or just
  $\Hom_R (\vectorspE,\vectorspF)$ or $\Hom (\vectorspE,\vectorspF)$
  if now confusion can arrise. 
\end{definition}

\begin{theorem}
  The topological vector spaces over a topological division ring $R$ as objects together 
  with their morphisms form an additive category which we denote by  $R\hyphen\category{TVS}$.
  More precisely, $R\hyphen\category{TVS}$ is a category enriched over the category of
  $R$-vector spaces where addition and scalar multiplication on the hom-spaces
  $\Hom (\vectorspE,\vectorspF)$  are given by
  \begin{align}
  \nonumber  
    +:&\:\Hom (\vectorspE,\vectorspF) \times \Hom (\vectorspE,\vectorspF) \to \Hom (\vectorspE,\vectorspF),\:
        (f,g) \mapsto f+g = \left(\vectorspE \ni v \mapsto f(v)+g(v)\in \vectorspF\right)\ , \\ \nonumber
    \cdot:&\:R  \times \Hom (\vectorspE,\vectorspF) \to \Hom (\vectorspE,\vectorspF),\:
        (r,f) \mapsto r \cdot f = \left(\vectorspE \ni v \mapsto r \cdot f(v) \in \vectorspF\right) \ .
  \end{align}
\end{theorem}

\begin{proof}
  Observe first that the identity map $\id_\vectorspE$ on a topological vector space $\vectorspE$
  is linear and continuous and so is the composition $g\circ f$ of two
  morphisms of topological vector spaces $f:\vectorspE \to\vectorspF$ and
  $g:\vectorspF \to\vectorspG$. Hence topological vector spaces over $R$ together with linear
  and continuous maps between them form a category.

  Next check that the hom-space $\Hom (\vectorspE,\vectorspF)$ is an abelian group.
  Associativity and commutativity of addition
  follow from the respective properties on $\vectorspF$. The zero element is the constant map
  $\vectorspE \to\vectorspF$, $v \mapsto 0$ and the inverse of a morphism $f:\vectorspE \to\vectorspF$
  is given by $-f:\vectorspE \to\vectorspF$,  $v \mapsto -f(v)$.
  Similarly one checks that multiplication by scalars on $\Hom (\vectorspE,\vectorspF)$ is associative
  and distributes from the left and from the right over addition since scalar multiplication on
  $\vectorspF$ has these properties. Finally, the unit of $R$ acts as identity on
  $\Hom (\vectorspE,\vectorspF)$ since it does so  on  $\vectorspF$. Hence $\Hom (\vectorspE,\vectorspF)$
  carries the structure of an $R$ left vector space.

  Composition of morphisms
  $\Hom (\vectorspE,\vectorspF) \times \Hom (\vectorspF,\vectorspG) \to \Hom (\vectorspE,\vectorspG)$,
  $(f,g)\to g \circ f $ is an $R$-bilinear map as the following equalities for 
  $f,f_1,f_2\in \Hom (\vectorspE,\vectorspF)$, $g,g_1,g_2 \in \Hom (\vectorspF,\vectorspG)$,
  $r\in R$, and $v\in \vectorspE$ show:
  \begin{align}
    \nonumber
    &(f \circ (g_1+g_2)) (v) = f( (g_1+g_2) (v) ) = f( g_1(v) +g_2(v) ) = \\ \nonumber
    & \quad = f\circ g_1(v) + f\circ g_2(v) =  (f\circ g_1+ f\circ g_2) (v)\ , \\ \nonumber
    &(f \circ (r g)) (v) = f( (rg) (v) ) = f(r g(v))  = r f( g(v)) =  (r(f\circ g)) (v)\ ,  \\ \nonumber
    &((f_1+f_2) \circ g) (v) = (f_1+f_2) ( g (v)) = f_1(g(v)) + f_2 (g(v)) = \\ \nonumber
    & \quad = f_1\circ g(v) + f_2\circ g(v) =  (f_1\circ g+ f_2\circ g) (v)\ , \\ \nonumber
    &((rf) \circ g) (v) = (rf)( g (v)) = r (f(g(v)))  = r (f \circ g (v)) =  (r(f\circ g)) (v) \ .
  \end{align}
  Hence $R\hyphen\category{TVS}$ is a category enriched over the category of $R$-vector spaces.
  In particular,  $R\hyphen\category{TVS}$ then is an additive category. 
\end{proof}

\begin{example}
  For every \tvs~$\vectorspE$ and non-zero element $t$ of the ground ring $R$ the
  map $\ell_t :\vectorspE \to \vectorspE$, $v \mapsto tv$ is an isomorphism of topological vector spaces
  by \Cref{thm:basic-properties-sets-maps-topological-vector-spaces} \ref{ite:translated-homotheties-homeomorphisms}.
\end{example}

\begin{propanddef}
  A linear map $f: \vectorspE \to \vectorspF$ between topological
  vector spaces over a valued division ring $(R,|\cdot|)$
  maps symmetric sets to symmetric sets and  circled sets to circled sets.
  If in addition $f$ is continuous, then $f$ is \emph{bounded} that means
  it  maps bounded subsets of $\vectorspE$ to bounded subsets of
  $\vectorspF$.   
\end{propanddef}

\begin{proof}
  Since by linearity $f(tv) = t f(v)$ for all $v\in \vectorspE$ and $t\in R$,
  $f (C)$ is symmetric (respectively circled) if the subset $C\subset \vectorspE$ is.

  To verify the second claim let $B\subset \vectorspE$ be bounded and
  $V\subset \vectorspF$ a zero neighborhood. Then $f^{-1} (V)$ is a zero neighborhood
  in $\vectorspE$ by continuity of $f$. Hence there exists an $r\in \R_{\geq 0}$
  such that $B \subset tf^{-1} (V)  $ for all $t\in R$ with $|t|\geq r$. By linearity
  of $f$ one obtains $f(B) \subset tV$ for all such $t$, so $f$ is bounded. 
\end{proof}

\begin{remark}
  By the proposition continuity of a linear map between topological vector spaces
  implies the map to be bounded. As we will see later in this monograph,
  the converse does in general not hold true unless the underlying
  topological vector spaces are for example normable. 
\end{remark}

\subsec{Normed real division algebras and local convexity}

\para
The major class of topological divison rings over which topological vector spaces are defined
is formed by valued division rings $(R,|\cdot|)$ which carry the structure of an
$\R$-algebra such that for all $r\in \R$ and $x\in R$ the equality
\[
         | rx | = |r|_\infty\cdot |x| 
\]
holds true. We will therefore given them a particular name and call them
\emph{normed real division algebras}. Note that the field of real numbers can be  embedded into
a normed real division algebra $R$ by the natural map $ \R \mapsto R$, $r \mapsto r\cdot 1$.  
Since $\R$ with its standard absolute value is archimedean, so is every normed real division algebra.
By the Frobenius theorem, \cite{FroLSBF}, there exist only three finite dimensional real division algebras,
namely the field of real numbers $\R$, the field of complex numbers $\C$, and the quaternions $\H$. 

\begin{definition}
  Under the assumption that $R$ is a normed real division algebra one calls a subset $C\subset \vectorspE$ of an $R$-vector space
  \begin{romanlist}
  \item
    \emph{convex} if $tv + (1-t)w \in C$ for all $v,w \in C$ and $t \in \R$ with $0\leq t \leq 1$,
  \item
    \emph{absolutely convex} if $rv + sw \in C$ for all $v,w \in C$ and $r,s \in R$ such that
     $|r|+|s|\leq 1$, 
  \item 
     a \emph{cone} if $tv \in C$ for all $v\in C$ and $t \in \R$ with $0\leq t \leq 1$. 
  \end{romanlist}
\end{definition}

\begin{lemma}
  Let $R$ be a normed real division algebra. A subset $C$ of an
  $R$-vector space $\vectorspE$ then is  absolutely convex if and only
  if it is circled and convex.  
\end{lemma}

\begin{proof}
   The claim is trivial when $C=\emptyset$, so we  assume that $C$ is nonempty. 

   Let $C$ be absolutely convex. Since $C$ contains at least one element $v$ 
   one has $0 = 0 \cdot v + 0 \cdot v \in C$. Hence 
   $ rv =  (1-|r|) \cdot 0  + r v \in C$ for all $v\in C$ and $r\in R$ with $|r|\leq 1$.
   So $C$ is circled. By definition of absolute convexity $C$ is convex.

   If $C$ is circled and convex, then it contains with elements $v,w$ also $r v + s w$ 
   if $|r| + |s|\leq 1$. To see this observe first that $\varrho v \in C$ and $\sigma w \in C$  
   where the elements $\varrho,\sigma \in R $ have been chosen so that 
   $|\varrho|=|\sigma|=1$, $r = |r|  \cdot\varrho$ and $s = |s|  \cdot\sigma$. Now if $|r| + |s| =0$, then
   $r v + s w = 0 \in C$ since $C$ is circled. If $|r| + |s| > 0$, then
   \[
    r v + s w = 
    (|r| + |s|) \left( \frac{|r|}{|r|+|s|} \varrho v  +  \frac{|s|}{|r|+|s|} \sigma w\right) \in C
   \]
   since $C$ is convex and circled. Hence $C$ is absolutely convex.
\end{proof}

\begin{lemma}
  A linear map $f:\vectorspE \to \vectorspF$ between vector spaces over a normed
  real divison algebra $R$ maps convex sets to convex sets,
  absolutely convex sets to absolutely convex sets, and cones to cones.
\end{lemma}

\begin{proof}
  This an immediate consequence of the linearity of $f$.
\end{proof}

\begin{lemma}
\label{thm:closure-interior-convex-set-convex}
  Let $\vectorspE$ be a \tvs~over a normed real division algebra $R$, let $C,D \subset \vectorspE$ be convex 
  and $r \in R$. Then the following holds true.
  \begin{romanlist}
  \item\label{ite:closure-interior-convex-set-convex}
    The closure $\closure{C}$ and the interior $\interior{C}$ are convex.
  \item\label{ite:sum-multiple-convex-set-convex}
    The sets $C + D$ and $rC$ are convex.
  \item
    If $C$ is absolutely convex, then so are $\closure{C}$  and $\interior{C}$.
  \item
    If $C$ is absolutely convex, then so is $rC$ for each $r \in R^\times$. 
  \end{romanlist} 
\end{lemma}

\begin{proof}
  We consider only the cases $C,D\neq \emptyset$ because otherwise the claim is trivial.
  \begin{adromanlist}
  \item
    Let $t\in \openint{0,1}$. Then $t \closure{C} + (1-t)\closure{C} \subset \closure{C} $
    by continuity of the map $\vectorspE \times \vectorspE \to \vectorspE$, $(v,w) \mapsto tv +(1-t)w$. 
    Hence $\closure{C}$ is convex. Now let $v,w$ be points of the interior of $C$ and $ z =tv + (1-t)w$.
    Then $z\in C$, and there exists a zero neighborhood $U$ such that $v+U \subset C$ and $w+U\subset C$. 
    Let $u\in U$ and compute
    \[
       z + u = tv + (1-t)w + tu + (1-t)u = t(v+u) +  (1-t)(w+u) \ .
    \]
    Since both $v+u$ and $w+u$ are elements of $C$ so is $z+u$ by convexity of $C$.
    Hence $z +U \subset C$ and $z$ lies in the interior of $C$.  
  \item
    If $v,w \in C$, $x,y\in D$ and $t\in \openint{0,1}$, then by convexity of $C$ and $D$
    \[
       t(v+x) + (1-t) (w+y) = \big( tv +(1-t) w \big) +  \big( tx +(1-t) y  \big) \in C + D \ . 
    \]
    Hence  $C+D$ is convex. Similarly,
    \[
       t (rv) + (1-t) (rw) = r \big(  t v + (1-t) w \big) \in r C \ ,
    \]
    so $rC$ is convex as well.
  \item
    Let $C$ be absolutely convex. If $\interior{C}\neq \emptyset$, then
    $0\in \frac 12 \interior{C} - \frac 12 \interior{C} \subset C$, hence $0 \in \interior{C}$.
    By \Cref{thm:closure-interior-circled-set-circled} and
    \ref{ite:closure-interior-convex-set-convex}
    the claim now follows.
  \item
    By \ref{ite:sum-multiple-convex-set-convex}, $rC$ is convex, so it remains to show
    that $rC$ is circled. Assume that $v\in rC$. Then $v = rw$ for a unique $w\in C$.
    Since $C$ is circled, $tw\in C$ for every $t\in R$ with $|t|\leq 1$.
    Hence $tv = r (tw) \in rC$ for such $t$ and $rC$ is circled.
  \end{adromanlist} 
\end{proof}

\begin{propanddef}\label{thm:absolutely-convex-hulls}
  The intersection of a non-empty family $(C_i)_{i\in I}$ of convex (respectively absolutely convex) subsets
  $C_i \subset \vectorspE$, $i\in I$ of a topological vector space $\vectorspE$ over a normed real division algebra $R$
  is convex (respectively absolutely convex).
  In particular, if $A \subset \vectorspE$ is a subset, then the sets
  \[
    \convHull A = \bigcap\limits_{A \subset B \subset \vectorspE \atop {B \text{ is convex}} }  B \quad \text{and} \quad
    \absconvHull A = \bigcap\limits_{A \subset B \subset \vectorspE \atop {B \text{ is absolutely convex}} }  B
  \]
  are convex and absolutely convex, respectively. The set $\convHull A$ is called the \emph{convex hull} of $A$
  and is the smallest convex set containing $A$. Similarly, $\absconvHull A$ is the smallest absolutely convex set
  containing $A$. It is called the  \emph{absolutely convex hull} of $A$.
  The \emph{closed convex hull} $\clconvHull A$ and the \emph{closed absolutely convex hull} $\clabsconvHull A$
  of $A$ are defined by  
  \[
    \clconvHull A = \bigcap\limits_{A \subset B = \closure{B} \subset \vectorspE \atop {B \text{ is convex}} }  B
    \quad \text{and} \quad
    \clabsconvHull A = \bigcap\limits_{A \subset B = \closure{B} \subset \vectorspE \atop {B \text{ is absolutely convex}} }  B \ .
  \]
  These sets have the property that $\clconvHull A$ is the smallest closed convex subset
  and $\clabsconvHull A$ the smallest closed absolutely convex subset of $\vectorspE$ containing $A$.
\end{propanddef}

\begin{proof}
  Let $C$ be the intersection $\bigcap\limits_{i\in I} C_i$ and assume that each $C_i$ is absolutely convex.
  Let $v,w\in C$ and $r,s \in R$ with $|r|+|s|\leq 1$. Then $v,w\in C_i$, hence  $rv + sw \in C_i$ for all $i\in I$. Therefore
  $rv +sw \in C$ and $C$ is absolutely convex. This argument also shows that $C$ is convex if all $C_i$ are convex.
  The rest of the claim follows as in the proof  of \Cref{thm:symmetric-circled-hulls}.
\end{proof}

\begin{remark}
  The proposition in particular entails that $A$ is convex if and only if $\convHull A = A$ and
  absolutely convex if and only if $\absconvHull A = A$. Analogously,
  $\clconvHull A = A$ if and only if $A$ is closed and convex,
  and $\clabsconvHull A = A$ if and only if $A$ is closed and absolutely convex. 
\end{remark}

\begin{lemma}\label{thm:representation-convex-absolutely-convex-hulls-closures}
  Let $A \subset \vectorspE$ be a non-empty subset of a \tvs~$\vectorspE$
  over a normed real division algebra $R$. Then
  \begin{align}
    \label{eq:convex-hull-expression}
    & \convHull A  = \left\{ \sum_{i=1}^k t_i v_i\in
    \vectorspE \bigmid k \in \gzN,\: v_1,\ldots v_k\in A,\: t_1\ldots ,t_k \in \R_{\geq 0}, \: \sum_{i=1}^k t_i = 1 \right\} 
      \ , \\
    \label{eq:absolutely-convex-hull-expression}
    & \absconvHull A  = \left\{ \sum_{i=1}^k r_i v_i\in
    \vectorspE \bigmid k \in \gzN,\: v_1,\ldots v_k\in A,\: r_1\ldots ,r_k \in R, \: \sum_{i=1}^k |r_i| \leq 1 \right\}
    \ .   
  \end{align}
  For the closed hulls one has  
  \[
    \clconvHull A = \overline{\convHull A} \quad \text{and} \quad \clabsconvHull A = \overline{\absconvHull A} \ .
  \]
  Finally, if $A$ is circled, then
  \[
     \absconvHull A = \convHull A  \ .
  \]
\end{lemma}

\begin{proof}
  By definition, the right hand side of Eq.~\eqref{eq:convex-hull-expression} is convex and contains $A$,
  hence it contains $\convHull A$. Conversely, one shows by induction on $k\in \gzN$ and convexity of $\convHull A$
  that each element of the form $\sum_{i=1}^k t_i v_i$ with $v_1,\ldots , v_k\in A$ and
  $t_1,\ldots ,t_k \in \R_{\geq 0}$  such that $\sum_{i=1}^k t_i = 1$ is in  $\convHull A$. This proves
  Eq.~\eqref{eq:convex-hull-expression}.
  The proof of Eq.~\eqref{eq:absolutely-convex-hull-expression} is similar. Observe that the right hand side of
  Eq.~\eqref{eq:absolutely-convex-hull-expression} is absolutely convex and contains $A$.
  Hence it contains $\absconvHull A$. An argument using induction on $k\in \gzN$ and absolute convexity of
  $\absconvHull A$ shows that each element of the form $\sum_{i=1}^k r_i v_i$ with $v_1,\ldots v_k\in A$ and
  $r_1\ldots ,r_k \in R$  such that $\sum_{i=1}^k |r_i| \leq 1$ is in  $\convHull A$.
  So Eq.~\eqref{eq:absolutely-convex-hull-expression} holds true as well. 
  The claim about the closed hulls is a consequence of \Cref{thm:closure-interior-convex-set-convex}.
  For the proof of the last claim it suffices to show that $\convHull A$ is circled if $A$ is.
  To this end let $v \in \convHull A$  and $r\in R$ with $|r|\leq 1$. Then
  one can write $v$ in the form $v = \sum_{i=1}^k t_i v_i$ with
  $v_1,\ldots, v_k\in A$ and $t_1, \ldots ,t_k \in \R_{\geq 0}$, where $\sum_{i=1}^k t_i = 1$.
  Hence $rv =  \sum_{i=1}^k t_i (rv_i)$, which is in $\convHull A$, since $rv_i\in A$ for all $i$ because
  $A$ is circled. 
\end{proof}

\begin{lemma}\label{thm:weighted-sum-convex-absolutely-convex-sets}
  Let $A \subset \vectorspE$ be a non-empty subset of a \tvs~$\vectorspE$
  over a normed real division algebra $R$. 
  \begin{romanlist}
  \item\label{ite:weighted-sum-convex-sets}
    If $A$ is convex and $t_1,\ldots, t_k\in \R_{\geq 0}$ with $k\in \gzN$, then
    \[
       \sum_{i=1}^k t_iA = \left( \sum_{i=1}^k t_i \right) A \ .
    \]
  \item
    If $A$ is absolutely convex and $r_1,\ldots, r_k\in R$ with $k\in \gzN$, then
    \[
       \sum_{i=1}^k r_iA = \left( \sum_{i=1}^k |r_i| \right) A \ .
    \]   
  \end{romanlist}
\end{lemma}

\begin{proof}
  \begin{adromanlist}
  \item
    Obviously $\sum_{i=1}^k t_iA \supset \left( \sum_{i=1}^k t_i \right) A$.
    Let us show the converse  inclusion. Without loss of generality we can
    assume that $t_i > 0$ for all $i$. Then $t = \sum_{i=1}^k t_i >0$,
    so, after division by $t$, we can reduce the claim to showing that
    $\sum_{i=1}^k t_iA \subset A$ for $t_1,\ldots , t_k\in \R_{>0}$ such that
    $\sum_{i=1}^k t_i =1$.
    But $\sum_{i=1}^k t_iA \subset \convHull A = A$ by
    \Cref{thm:representation-convex-absolutely-convex-hulls-closures} and
    convexity of $A$. 
  \item
    Since by absolute convexity $r_i A = |r_i|A$ for $i=1,\ldots,k$, the claim
    follows from \ref{ite:weighted-sum-convex-sets}.
  \end{adromanlist}
\end{proof}

\begin{lemma}\label{thm:real-absorbance-implies-arbsorbance-finite-dimensional-divison-algebra}
  Let $\fldK$ be one of the division rings $\C$ or $\H$ with their standard absolute values
  and let $\vectorspE$ be a vector space over $\fldK$. Then a convex subset $C\subset \vectorspE$ is
  absorbent in $\vectorspE$ if and only if it is absorbent in the realification $\vectorspE^\R$.
\end{lemma}

\begin{proof}
  It suffices to show the non-trivial direction. So assume that $C$ is convex and absorbent in the
  realification $\vectorspE^\R$. Denote by $u_1,\ldots , u_n$  the standard basis of $\fldK$
  over $\R$ with $n=2$ or $n=4$ depending on $\fldK$. In particular this means $u_1 =1$.
  For given $v\in\vectorspE$ there now exists $t\in\R_{\geq 0}$ such that
  \[
    \pm \frac{1}{u_1} v , \ldots  , \pm \frac{1}{u_n} v \in rC \quad \text{for all } r\geq t \ .
  \]   
  Without loss of generality we can assume $t\geq 1$. Let $z\in \fldK$ with $|z| \geq n t$. 
  Then the vectors
  $c_1 = \sgn z_1\frac{n}{|z| \, u_1} v, \ldots , c_n = \sgn z_n\frac{n}{|z| \, u_n} v$
  are elements of $C$. By convexity of $C$ and since $0\in C$ one has
  $ \frac{|z_1|}{|z|} c_1 , \ldots , \frac{|z_n|}{|z|} c_n  \in C$. Again by convexity one concludes
  \[
    \frac{1}{z} v = \sum_{i=1}^n \frac{z_i}{|z|^2 \, u_i} v =
    \sum_{i=1}^n  \frac{|z_i|}{n |z|} c_i \in C \ . 
  \]
  Hence $C$ is absorbing and the claim is proved.   
\end{proof}

\begin{definition}
  A topological vector space $\tvsE$ over a normed real division algebra $R$ for which
  Axiom \hyperref[axiom:tvs-local-convexity]{LCVS} below holds true is called a
  \emph{locally convex topological vector space}, a \emph{locally convex vector space} or shortly a
  \emph{locally convex \tvs}.
  \begin{axiomlist}[\hspace{1.8em}]
  %\setcounter{enumi}{2}
  \item[\textup{(}{\sffamily LCVS}\textup{)}\!]
  \label{axiom:tvs-local-convexity} 
    The vector space topology on $\tvsE$ has a base consisting of convex sets. 
  \end{axiomlist}
\end{definition}

\begin{remark}
  For better readability, we often say \emph{locally convex topology} instead of \emph{locally convex vector space topology}.
\end{remark}

\begin{proposition}
  The locally convex topological vector spaces over a normed real division algebra $R$ together with the continuous
  linear maps between them form a full subcategory of the category $R\text{-}\category{TVS}$ of topological
  $R$-vector spaces. It is denoted $R\text{-}\category{LCVS}$.
\end{proposition}
\begin{proof}
  This is clear by definition.
\end{proof}

\begin{propanddef}\label{thm:filter-base-absorbing-absolutely-convex-sets-generating-locally-convex-topology}
  The filter of zero neighborhoods of a locally convex topological vector space $\tvsE$ over a
  normed real divison algebra $R$ has a filter base $\base$ with the following properties:
  \begin{romanlist}
  \item\label{ite:lctvs-base-element-containing-sum-base-elements}
    For each $V \in \base$ there exists $U\in\base$ such that $U+U \subset V$.
  \item\label{ite:lctvs-base-elements-circled-balanced}
    Every element of $\base$ is a \emph{barrel} that means is absolutely convex, closed and absorbing.
  \item\label{ite:lctvs-base-element-containing-shrunk-stretched-base-element}
    Let  $r\in R^\times$. Then $V\in \base$ if and only if $rV \in \base$. 
  \end{romanlist}
  Conversely, if  $\base$ is a filter base on an $R$-vector space $\vectorspE$ such that
  \ref{ite:lctvs-base-element-containing-sum-base-elements} holds true and such that each
  element of $\base$ is absolutely convex and absorbing, then there exists a unique locally
  convex topology on $\vectorspE$  such that $\base$ is a neighborhood base
  of the origin. It is the coarsest among all translation invariant  topologies
  for which  $\base$ is a set of zero neighborhoods and is called the \emph{locally convex topology generated} or
  \emph{induced by} $\base$.
\end{propanddef}

\begin{proof}
  Let $\tvsE$ be a locally convex \tvs. Let $\base$ be the collection of all barrels 
  which are at the same time zero neighborhoods. Let $V$ be an element of $\nbhdfilter_0$,
  the filter of zero neighborhoods. Since $\tvsE$ is \textsf{(T3)} by
  \Cref{thm:basic-properties-sets-maps-topological-vector-spaces}, there exists a closed
  zero neighborhood $V_a$ such that $V_a \subset V $.  By local convexity of $\tvsE$ there exists
  a convex zero neighborhood $V_b$ with $V_b\subset V_a$. By
  \Cref{thm:topological-vector-space-zero-neighborhood-filter-base-circled-absorbing} there exists a circled
  zero neighborhood $V_c$ with $V_c\subset V_b$.  The closed convex hull $U = \clconvHull V_c$ then is 
  a barrel contained in $V$. Since it is a zero neighborhood it is an element of $\base$, and $\base$
  is a filter base of $\nbhdfilter_0$.
  This proves \ref{ite:lctvs-base-elements-circled-balanced}.
  
  To verify \ref{ite:lctvs-base-element-containing-sum-base-elements}, let $V \in \base$ and observe that
  by continuity of addition there exist zero neighborhoods $U_1$ and $U_2$ such that $U_1 + U_2 \subset V$.
  Choose $U\in \base $ such that $U \subset U_1\cap U_2$. Then $U + U \subset V$.

  Claim \ref{ite:lctvs-base-element-containing-shrunk-stretched-base-element} holds true since multiplication
  by an element $r\in R^\times$ is a homeomorphism which preserves circled and convex sets.

  The remaining claim follows immediately from
  \Cref{thm:topological-vector-space-zero-neighborhood-filter-base-circled-absorbing}
  and the observation that a real division algebra is archimedean.
\end{proof}

\begin{corollary}
  Let  $\mathscr{S}$ be a non-empty set of absolutely convex and absorbent
  subsets of a vector space $\vectorspE$ over a normed real divison algebra $R$.
  Then the set
  \[
    \base = \Big\{ r\, \bigcap\limits_{B\in\mathscr{F}} B \in\powerset{\vectorspE} \bigmid
    \mathscr{F}\in \powersetfin{\mathscr{S}},\: \mathscr{F} \neq \emptyset \:\: \& \:\: r\in R^\times 
    \Big\}
  \]
  consists of absolutely convex and absorbent subsets of $\vectorspV$ and 
  is a base of the filter of zero neighborhoods of a locally convex topology
  $\topology$ on $\vectorspE$ uniquely determined by that property.
  This topology is the coarsest among all vector space  topologies
  for which  $\mathscr{S}$ is a set of zero neighborhoods. The topology $\topology$ is called the
  \emph{locally convex topology generated} or \emph{induced} by $\mathscr{S}$. 
\end{corollary}

\begin{proof}
  The intersection of finitely many absolutely convex and absorbing
  sets is non-empty and again absolutely convex and absorbing by
  \Cref{thm:intersection-scalar-multiples-absorbing-sets-absorbing} \ref{ite:intersection-absorbing-sets-absorbing}
  and \Cref{thm:absolutely-convex-hulls}. 
  By \Cref{thm:intersection-scalar-multiples-absorbing-sets-absorbing}
  \ref{ite:scalar-multiple-absorbing-set-absorbing} and
  \Cref{thm:closure-interior-convex-set-convex}, the scalar multiple of an absolutely convex and absorbing set
  again has these properties whenever the scalar is invertible. 
  Hence each element of $\base$ is absolutely convex and absorbing. Given two elements $C,D\in \base$ there 
  exist non-empty $\mathscr{F},\mathscr{G}\in \powersetfin{\mathscr{S}}$ and $r,s\in R^\times$
  such that $ C = r \bigcap\limits_{B \in \mathscr{F}} B $ and
  $D = s \bigcap\limits_{B \in \mathscr{G}} B$. Without loss of generality
  one can assume that $|r|\leq |s|$. Then
  $A= r  \bigcap\limits_{B \in \mathscr{F}\cup  \mathscr{G}} B \in \base$
  and $A  = C \cap rs^ {-1}D  \subset C \cap D$ since $D$ is balanced and $|rs^{-1}| \leq 1$.  
  Hence $\base$ is a filter base consisting of absolutely convex and absorbent
  sets. Moreover, $\frac 12 C + \frac 12 C\subset C $ for every $C \in \base$
  by absolut convexity.
  By Proposition \ref{thm:filter-base-absorbing-absolutely-convex-sets-generating-locally-convex-topology}
  the filter base $\base$ therefore generates
  a unique locally convex topology $\topology$ for which  $\base$ is a base of the filter of zero
  neighborhoods. Moreover, $\topology$ is the coarsest translation invariant topology so that $\base$ is a
  set of zero neighborhoods. This implies in particular that $\mathscr{S}$ is a set of zero
  neighborhoods for $\topology$. Now let $\topology'$ be a vector topology such that each
  element of $\mathscr{S}$ is a zero neighborhood. Then finite intersections of elements of 
  $\mathscr{S}$ are zero neighborhoods with respect to $\topology'$ and therefore also
  all elements of $\base$. Since $\topology'$ is translation invariant one concludes
  that  $\topology$ is coarser than $\topology'$ and the claim is proved.  
\end{proof}