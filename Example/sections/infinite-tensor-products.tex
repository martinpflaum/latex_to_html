% Copyright 2020 Markus J. Pflaum, licensed under CC BY-NC-ND 4.0
% main author: 
%   Markus J. Pflaum 
%
\section{Infinite tensor products}
\label{sec:infinite-tensor-products}

\para
Infinite tensor products of Hilbert spaces were introduced by  \cite{vNeuIDP}. They were motivated by mathematical physics
where one needs to describe quantum systems with infinitely many degrees of freedom, see e.g.~\cite{EmcAMSMQFT,BraRobOAQSM2}.
The original construction of infinite tensor products was generalized to von Neumann  and $C^*$-algebras
by \cite{GuiPTIRRA}, \cite{BlaITPC*A}, and others. Meanwhile, the topic has been studied in quite some
detail in the operator algebra literature, see e.g.~\cite{NakITPvNAI,NakITPvNAII,StoITPvNA}. 
A purely algebraic or better categorical approach allowing the construction of infinite tensor products of modules over
a given commutative ring has been given in \cite[Sec.~III.10]{CheFCA}. The work \cite{NgGIATP} is also in that spirit.
We will essentially follow \cite{CheFCA} and construct the infinite tensor product as a module
universal with respect to multilinear maps. First we present the main algebraic construction, then we explain some of
the subtleties which distinguish infinite from finite tensor products, and finally we construct infinite Hilbert
tensor products and infinite tensor products of $C^*$-algebras. 

\para\label{para:canonical-projections-embeddings-multilinear-maps}
Let $R$ be a commutative ring and $(M_i)_{i\in I}$ a possibly infinite family of $R$-modules.
Consider $\prod_{i\in I} M_i$, the product of the family $(M_i)_{i\in I}$  within the category
of $R$-modules. For each $j\in I$ let $\pi_j : \prod_{i\in I} M_i \to M_j $ denote the natural
projection onto the $j$-th factor and $\iota_j :M _j \hookrightarrow  \prod_{i\in I}M_i$ the
uniquely determined natural embedding such that 
\[
   \pi_j\circ \iota_i =  
   \begin{cases}
       \id_{M_i} & \text{for}\enspace i=j \enspace\text{and} \\
       0 & \text{else} .  
   \end{cases}
\]
Given an $R$-module $N$  one then understands by a \emph{multilinear map} from $ \prod_{i\in I}M_i$
to $N$ a map $f:  \prod_{i\in I}M_i \to N$
such that for each $j\in I$ and $x\in \prod_{i\in I} M_i$ with $\pi_j (x)=0$ the map $M_j\to N$, $m\mapsto f(\iota_j(m)+x)$ is linear.
The set of multilinear maps from $\prod_{i\in I} M_i$ to $N$ will be denoted by
$\mlinOps \big(\prod_{i\in I} M_i,N\big)$. It carries a natural structure of an $R$-module
given by pointwise addition of multilinear maps and pointwise action of a scalar on a
multilinear map that is by  
\[
  f+ g =  \left( \prod_{i\in I} M_i \ni x \mapsto f(x) + g(x) \in N \right) \quad\text{and}\quad
  r f =  \left( \prod_{i\in I} M_i \ni x \mapsto rf(x)\in N \right) 
\]
for all $f,g \in \mlinOps \big(\prod_{i\in I} M_i,N\big)$ and $r\in R$.
Since for $j\in I$ and $x\in \prod_{i\in I} M_i$ with $\pi_j(x)= 0$ the maps
$M_j\to N$, $m \mapsto (f+g) (\iota_j(m) + x) = f (\iota_j(m) + x) + g (\iota_j(m) + x)$
and $M_j\to N$, $m \mapsto rf (\iota_j(m) + x) $ are linear by assumption on $f$ and $g$,
the maps $f+g$ and $rf$ are multilinear again, so $ \mlinOps\big(\prod_{i\in I} M_i,N\big)$ is an
$R$-module indeed with zero element the constant function mapping to $0\in N$. 

\begin{remarks}
  Before proceeding further let us make several explanations concerning the notation used.
  \begin{environmentlist}
  \item
    The space of multilinear maps $ \mlinOps \big(\prod_{i\in I} M_i,N\big)$ actually depends
    on the family $(M_i)_{i\in I}$ and the $R$-module $N$, so in principle one should  write
    $ \mlinOps \big((M_i)_{i\in I},N\big)$ instead of $ \mlinOps \big(\prod_{i\in I} M_i,N\big)$.
    Nevertheless we stick to the latter notation since it is closer to standard notation for
    linear maps and since it will not lead  to any confusion.
  \item
    In case the index set $I$ has just two elements $i_1,i_2$, one calls a multilinear map
    $\prod_{i\in I}M_i = M_{i_1}\times M_{i_2} \to N$ a \emph{bilinear map}. If the cardinality of
    $I$ is $3$, one sometimes calls a multilinear map $\prod_{i\in I}M_i \to N$ a
    \emph{trilinear map}. 
    
  \item
     In the following, when saying that $(I_a)_{a\in A}$ is a partition of the set $I$ we mean that
  each $I_a$ is a non-empty subset of $I$, that $I_a   \cap I_b =\emptyset $
  for $a \neq b$ and that $\bigcup_{a \in A} I_a = I$. The empty family is regarded as
  a partition of the empty set.
    
  \item We will frequently use in this section the same symbol for 
  maps with the same ``universal'' properties  despite those maps might be  strictly speaking different. 
  For example, $\pi_k$ will stand for the canonical projections
  $\prod_{i\in I} M_i \to M_k$ and $\prod_{j\in J} M_j \to M_k$ whenever $k\in J \subset I$. 
  Likewise we use the same notation for the two canonical embeddings 
  $M_k \hookrightarrow \prod_{i\in I} M_i $ and $M_k \hookrightarrow \prod_{j\in J} M_j $
  defined in \ref{para:canonical-projections-embeddings-multilinear-maps} and denote them both by $\iota_k$.

  \end{environmentlist}
\end{remarks}

\begin{lemma}[cf.~{\cite[Sec.~III.10, Lemma 1 \& 2]{CheFCA}}]\label{thm:construction-multilinear-maps-composition}
  Assume that $(M_i)_{i\in I}$ is a family of $R$-modules, $N$ an $R$-module, and
  $f : \prod_{i\in I}M_i \to N$ a mutilinear map.
  \begin{romanlist}
  \item
    If $g :N \to N^\prime$ is an $R$-module map, then
    $g\circ f :  \prod_{i\in I}M_i \to N^\prime$ is multilinear.
  \item
    Let $J\subset I$ be non-empty, $y= (y_i)_{i\in I\setminus J}$ an element of the product
    $\prod_{i\in I\setminus J} M_i$, and $ \iota_{J,y} :  \prod_{j\in J} M_j \to \prod_{i\in I} M_i$ the unique map
    such that for all $x = (x_j)_{j\in J} \in  (M_j)_{j\in J}$ and $k\in I$
    \[
      \pi_k \circ \iota_{J,y}\, (x) = 
      \begin{cases}
        x_k & \text{for } k\in J , \\
        y_k & \text{for } k \in I\setminus J .
      \end{cases}
    \]
    Then the composition $f\circ \iota_{J,x} : \prod_{j\in J} M_j \to N$ is multilinear.
 \item\label{ite:multilinearity-composition-multilinear-map-product-multilinear-map}
   Let $(I_a)_{a \in A}$ be a partition of the index set $I$ which is assumed to be non-empty.
   Let $(N_a)_{a \in A}$ be a family of $R$-modules, $(g_a)_{a \in A}$
   a family of multilinear maps $g_a : \prod_{i \in I_a} M_i \to N_a$, and
   $h :  \prod_{a \in A}N_a \to N$ multilinear. Define
   $g: \prod_{i\in I} M_i \to  \prod_{a \in A} N_a $ as the unique map
   such that 
   \[
      \pi_b \circ g = g_b \circ \pi_{I_b} \quad 
      \text{for } b \in A  ,
    \]
    where $\pi_J$ for $J\subset I$ as on the right side stands for the
    projection $\pi_J : \prod_{i\in I} M_i \to  \prod_{j \in J} M_j$
    uniquely determined by $\pi_j \circ \pi_J =\pi_j$ for all $j\in J$. 
   Then the  composition $h \circ g : \prod_{i\in I} M_i \to N$ is multilinear.
  \end{romanlist}
\end{lemma}

\begin{proof}
  \begin{adromanlist}
  \item
    Let $j\in I$ and $x\in \prod_{i\in I} M_i$ with $\pi_j (x)=0$.
    By multilinearity of $f$ and linearity of $g$, the map $M_j\to N^\prime$,
    $m\mapsto g f (\iota_j(m)+x)$ then has to be linear, hence $g\circ f$ is multilinear.
  \item
    Let $j \in J$ and $x\in \prod_{i\in J} M_i$ with $\pi_j (x)= 0$.
    % Define $\widetilde{x}$ as the unique element of $\prod_{i\in I}M_i$ such that
    % \[
    %    \pi_k (\widetilde{x}) =
    %    \begin{cases}
    %      \pi_k(x)  &\text{for } k \in J,\\ 
    %      y_k &\text{for } k \in I\setminus J . 
    %    \end{cases}
    % \]
    Then $\pi_j(\iota_{J,y}(x))=0$ and $f \iota_{J,y} (\iota_j(m)+x) =
    f(\iota_j(m) + \iota_{J,y}(x)$ for all $m\in M_j$ by  construction of $\iota_{J,y}$.
    Hence the map $M_j\to N$, $m\mapsto f \iota_{J,y} (\iota_j(m)+x)$ is linear by
    multilinearity of $f$. This proves that $f\circ \iota_{J,y}$ is multilinear.
  \item
    Given $j\in I$ let $b$ be the unique element of $A$  such that $j\in I_b$.
    Assume  that $x\in \prod_{i\in I} M_i$ with $\pi_j (x)=0$.
    By construction one has $\pi_j(\pi_{I_b} (x))=0$.
    Now let $y \in \prod_{a \in A}N_a$ such that
    \[
       \pi_a (y) =
       \begin{cases}
         0 & \text{for } a = b , \\
         g_a \pi_{I_a} (x) & \text{for } a  \neq b.
       \end{cases}
    \]
    One then obtains for $m\in M_j$
    \[
    \pi_a g ( \iota_j(m) + x)= 
    \begin{cases}
      g_b \pi_{I_b} (\iota_j(m)+x) = g_b (\iota_j (m) + \pi_{I_b} (x)) &\text{for }a = b, \\
      g_a \pi_{I_a} (x) =  \pi_a (y)  & \text{for } a  \neq b .
    \end{cases}
    \]
    Hence 
    \[
      h g (\iota_j(m)+ x) =
      h \big( \iota_b \big( g_b ( \iota_j(m) + \pi_{I_b} (x) \big) + y \big) \ ,
    \]
    and the map $M_j\to N$, $m\mapsto h g (\iota_j(m)+ x)$ is linear as the composition of two linear maps. 
  \end{adromanlist}
\end{proof}

\begin{lemma}\label{thm:associator-cartesian-product}
  Assume to be given a non-empty family of $R$-modules $(M_i)_{i\in I}$ and a partition 
  $(I_a)_{a \in A}$ of the index set $I$. Then there exists a natural ismorphism
  \[
    \kappa_{I,A}:  \prod_{i \in I} M_i \to \prod_{a \in A}  \prod_{i \in I_a} M_i
  \]
  uniquely determined by the condition that $\pi_a \circ \kappa_{I,A} = \pi_{I_a}$ for all
  $a\in A$.
  % where $\pi_a$ on the right hand side stands for the unique projection
  % $\pi_a : \prod_{i \in I} M_i \to \prod_{i \in I_a} M_i$ which satisfies
  % $\pi_k \circ \pi_a = \pi_k$ for all $k\in I_a$.
\end{lemma}
\begin{proof}
  By the universal property of the product the $R$-module map 
  $ \kappa = \kappa_{I,A}:  \prod_{i \in I} M_i \to \prod_{a \in A}  \prod_{i \in I_a} M_i$
  exists and is uniquely determined by the requirement  that $\pi_a \circ \kappa_{I,A} = \pi_{I_a}$
  for all $a\in A$. Naturality also follows from the universal property of the product.
  It remains to show that $\kappa$ is an isomorphism. By construction,
  $\pi_i(x) = \pi_i \pi_a \kappa (x) =0$ for all $i\in I$ and $a(i)\in A$ such that $i\in I_{a(i)}$,
  hence $x=0$.
  So $\kappa$ is injective. It is also surjective. To see this pick $x_a \in \prod_{i \in I_a} M_i$
  for each $a\in A$. With $a(i)$ for $i\in I$ defined as before put
  $x = \big( \pi_i (x_{a(i)} )\big)_{i\in I}$. Then, by construction,
  $\pi_i \pi_{a} \kappa (x) = \pi_i \pi_{a} (x) = \pi_i (x) = \pi_i (x_{a})$ for all $a\in A$ and
  $i\in I_a$,
  hence $\big(\pi_a \kappa (x) \big)_{a \in A} = (x_a)_{a \in A}$
  and $\kappa$ is surjective. 
\end{proof}

\begin{proposition}[Exponential law for multilinear maps]
\label{thm:exponential-law-multilinear-maps}
  Let $(M_i)_{i\in I}$ be a family of $R$-modules over a commutative ring $R$,
  $N$ an $R$-module, 
  and assume that $J \subset I$ is a non-empty subset such that the complement
  $K = I\setminus J$ is also non-empty. Then the map
  \begin{equation*}
  \begin{split}
    \eta_{I,J}: \:&
    \mlinOps \left( \prod_{j\in J} M_j , \mlinOps \left(\prod_{k\in K} M_k ,N \right)\right)
    \rightarrow \mlinOps \left( \prod_{i\in I} M_i, N \right), \\
    & \hspace{1em} f \mapsto \left( \prod_{i\in I} M_i\ni (x_i)_{i\in I} \mapsto
    f\big( (x_j)_{j\in J} \big)\left( (x_k)_{k\in K} \right)  \in N \right) 
  \end{split}
  \end{equation*}
  is an isomorphism which is natural in $(M_i)_{i\in I}$ and $N$.
\end{proposition}

\begin{proof}
  
  We first show that $\eta= \eta_{I,J} $ is linear. To this end let\\
  $f,g \in \mlinOps \left( \prod_{j\in J} M_j , \mlinOps \left(\prod_{k\in K} M_k ,N \right)\right)$ and $r\in R$.
  Then, for all $x = (x_i)_{i\in I} \in \prod_{i\in I} M_i$,
  \begin{equation*}
  \begin{split}
    \big( \eta (f & +g)\big) (x)  = \big(f+g\big) \big( (x_j)_{j\in J} \big)\left( (x_k)_{k\in K} \right)
    = \big( f((x_j)_{j\in J}) + g ((x_j)_{j\in J})\big) \left( (x_k)_{k\in K} \right) = \\
    & = f((x_j)_{j\in J})  \left( (x_k)_{k\in K} \right) +  g ((x_j)_{j\in J}) \left( (x_k)_{k\in K} \right) =
     \big( \eta f \big) (x) + \big( \eta g \big) (x) =  \big( \eta f + \eta g \big) (x) 
  \end{split}
  \end{equation*}
  and
  \begin{equation*}
  \begin{split}
    \big( \eta (rf)\big) (x) & = (rf) ((x_j)_{j\in J})  \left( (x_k)_{k\in K} \right) =
    \big( r f((x_j)_{j\in J})\big) \left( (x_k)_{k\in K} \right) =
    r \big( f((x_j)_{j\in J})\left( (x_k)_{k\in K} \right)\big) = \\
    & = r \big( \eta f (x)\big) =  \big( r (\eta f ) \big) (x) \ .
  \end{split}
  \end{equation*}
  Hence $\eta$ is an $R$-module map.

  Next we show that $\eta$ is an isomorphism by constructing an inverse.
  Given $f \in \mlinOps \big( \prod_{i\in I} M_i, N \big)$ we define
  $f^\sharp : \mlinOps \big( \prod_{j\in J} M_j \big) \to \mlinOps \big(\prod_{k\in K} M_k ,N \big)$ by the requirement that 
  \[
      f^\sharp (y) (z) = f (x_{y,z}) \quad \text{for all}\enspace y = (y_j)_{j\in J}\enspace \text{and} \enspace z = (z_k)_{k \in K} \ ,
  \]
  where $x_{y,z}$ is the  element of $\prod_{i\in I}M_i$ uniquely determined by
  \[
    \pi_i (x_{y,z}) =
    \begin{cases}
      y_i & \text{for}\enspace i \in J , \\
      z_i & \text{for}\enspace i \in K .   
    \end{cases}
  \] 
  One thus obtains an $R$-module map 
  \[
    (-)^\sharp_{I,J} :
    \mlinOps \left( \prod_{i\in I} M_i, N \right) \to
    \mlinOps \left( \prod_{j\in J} M_j , \mlinOps \left(\prod_{k\in K} M_k ,N \right)\right),\quad f \mapsto f^\sharp
  \]
  which by construction is inverse to $\eta_{I,J}$. 

  Naturality of $\eta_{I,J}$ in $(M_j)_{j\in J}$ and $N$ is clear by definition. 
\end{proof}

\begin{definition}
  Let $(M_i)_{i\in I}$ be a family of $R$-modules over a commutative ring $R$. By a
  \emph{tensor product} of  $(M_i)_{i\in I}$ one understands an $R$-module $\bigotimes_{i\in I}M_i$
  together with a multilinear map
  $\tau : \prod_{i\in I}M_i \to \bigotimes_{i\in I}M_i$ such that the following universal property is fulfilled:
  \begin{axiomlist}[\hspace{2.5em}]
  \item[\textup{\sffamily (ITensor)}]
   \label{axiom:hilbert-tensor-product}
   For every $R$-module $N$ and every multilinear map 
   $f : \prod_{i\in I}M_i \to N$ there exists a unique $R$-module map
   $\overline{f}: \bigotimes_{i\in I}M_i \to N$ 
   such that the diagram 
   \begin{displaymath}
   \begin{tikzcd}
       \prod\limits_{i\in I}M_i  \ar[d,"\tau",swap] \ar[r,"f"]  & N \\
       \bigotimes\limits_{i\in I}M_i \ar[ru,"\overline{f}",swap]
   \end{tikzcd}
   \end{displaymath}
   commutes.
 \end{axiomlist} 

 The linear map $\overline{f}$ making the diagram comute will sometimes be called the \emph{linearization}
 of the multilinear map $f$.
   
 Given a tensor product
 $\big( \bigotimes_{i\in I}M_i,\tau\big)$, we will usually denote the image of an element 
 $(x_i)_{i\in I} \in \prod_{i\in I}M_i$
 under the map $\tau$ by $\otimes_{i\in I} x_i$. 
\end{definition}

\begin{remarks}
  \begin{environmentlist}
  \item
    Strictly speaking, a tensor product of a family $(M_i)_{i\in I}$ of $R$-modules is a pair
    $\big( \bigotimes_{i\in I}M_i,\tau\big)$ having the above properties. By slight abuse of language, one
    usually denotes a tensor product just by its first component, the $R$-module $\bigotimes_{i\in I}M_i$.
    When helpful for clarity, the associated map $\tau: \prod_{i\in I}M_i \to \bigotimes_{i\in I}M_i$
    will be denoted by $\tau_{(M_i)_{i\in I}}$ or by $\tau_I$. 
  \item   
    In the case where the index set $I$  of the family $(M_i)_{i\in I}$ is infinite, one
    sometimes calls $\bigotimes_{i\in I}M_i$ an \emph{infinite tensor product}. 
  \end{environmentlist}
\end{remarks}



\begin{theorem}\label{thm:construction-fundamental-properties-infinite-tensor-product}
  Let $(M_i)_{i\in I}$ be a family of $R$-modules over a commutative ring $R$. Then the following
  holds true.
  \begin{romanlist}
  \item
    A tensor product $\bigotimes_{i\in I}M_i$ of the family $(M_i)_{i\in I}$ exists and is
    unique up to isomorphism. 
    If $I$ is the empty set, then $\bigotimes_{i\in I}M_i = R$, if $I$ contains a single element $i_\circ$,
    then  $\bigotimes_{i\in I}M_i =M_{i_\circ}$.   
  \item
    If $(N_i)_{i\in I}$ is a second family of $R$-modules and $(f_i)_{i\in I}$ a family
    $R$-module maps $f_i :M_i \to N_i$, then there exists a unique linear map 
    $\bigotimes_{i\in I}f_i: \bigotimes_{i\in I}M_i \to \bigotimes_{i\in I}N_i$ making
    the  diagram
    \begin{displaymath}
    \begin{tikzcd}
       \prod\limits_{i\in I}M_i  \ar[d,"\tau",swap] \ar[r,"f"] & \bigotimes\limits_{i\in I}N_i   \\
       \bigotimes\limits_{i\in I}M_i \ar[ru,"\bigotimes\limits_{i\in I}f_i",swap]
   \end{tikzcd}
   \end{displaymath}
   commute, where $f:\prod_{i\in I}M_i\to \bigotimes_{i\in I}N_i$ is the multilinear map
   $(x_i)_{i\in I}\mapsto \otimes_{i\in I} f_i(x_i)$.
 \item
   Let $J\subset I$ be a finite non-empty subset set such that  $M_j$ is isomorphic  to $R$ for all $j\in J$.
   Denote for each $j\in J$ by $1_j$ the image of the unit $1\in R$ under the isomorphism $R\cong M_j$
   and by $1_J$ the family $(1_j)_{j\in J}$. Moreover, for every family $y =(y_j)_{j\in J}$
   let $\iota_{J,y} : \prod_{i\in I\setminus J} M_i \to \prod_{i\in I}M_i$ be the map which associates
   to $x\in\prod_{i\in I\setminus J} M_i$ the family $ (x_i)_{i\in I}$ such that $x_i =\pi_i (x)$ for
   $i \in I\setminus J$ and  $x_i = y_i$ for $i \in J$. Then the linearization
   $\overline{\iota}_{J,1_J}: \bigotimes_{i\in I\setminus J}M_i \to  \bigotimes_{i\in I}M_i $ of the multilinear map
   $\tau_I \circ \iota_{J,1_J}: \prod_{i\in I\setminus J}M_i \to  \bigotimes_{i\in I}M_i$ is an isomorphism. 
  \end{romanlist}
\end{theorem} 

\begin{proof}
\begin{adromanlist}
\item
  By its universal property, the tensor product of the family $(M_i)_{i\in I}$ is uniquely determined
  up to isomorphism. Hence it remains to show the existence of the tensor product.
  To this end consider the free $R$-module  % $R^{(\prod_{i\in I}M_i)}$
  over the set $\prod_{i\in I}M_i$ and denote it by $F$. Let  $\delta: \prod_{i\in I}M_i \hookrightarrow F$ be
  the canonical injection  
  %Identify an element $(x_i)_{i\in I}\in\prod_{i\in I}M_i$
  %with its image in $F$ under the canonical injection $\iota : \prod_{i\in I}M_i \hookrightarrow F$
  and $U$ be the submodule of $F$ spanned by the elements
  \[
    \delta \big( \iota_j (r y_j + z_j) + (x_i)_{i\in I} \big) - r \delta \big( \iota_j (y_j) + (x_i)_{i\in I}\big)
    - \delta \big( \iota_j (z_j) + (x_i)_{i\in I}\big) \ ,
  \]
  where $j\in I$, $y_j , z_j \in M_j$, $r \in R$,  and $(x_i)_{i\in I} \in \pi_j^{-1} (0)$. Then put
  $\bigotimes_{i\in I}M_i = F/U$ and define $\tau$ as the composition of the canonical projection
  $\pi : F \to \bigotimes_{i\in I}M_i$  with $\delta : \prod_{i\in I}M_i \to F$. By construction, $\tau$ is multilinear.
  Assume that $N$ is an $R$-module and $f : \prod_{i\in I}M_i \to N$ is a multilinear map. By the universal property of
  free $R$-modules, $f$ lifts to a unique $R$-linear map $f^\prime : F \to N$ such that $f = f^\prime \circ \delta$.
  By multilinearity of $f$, the map $f^\prime$ vanishes on the submodule $U$, hence descends to an $R$-linear
  $\overline{f}: \bigotimes_{i\in I}M_i \to N$ such that $f^\prime = \overline{f} \circ \pi$.
  Hence  $f = f^\prime \circ \delta = \overline{f} \circ \pi  \circ \delta  = \overline{f} \circ \tau$.
  By surjectivity of $\delta$ and uniqueness of $f^\prime$, $\overline{f}$ is the unique $R$-linear map satisfying
  $f = \overline{f} \circ \tau$. Hence $\big( \bigotimes_{i\in I}M_i,\tau\big)$ is a tensor product of the family
  $(M_i)_{i\in I}$.
  
  In case $I=\emptyset$, the cartesian product $\prod_{i\in I}M_i$ is final in the category of sets, hence 
  consists of only one element $\star$ only. This means in particular that 
  for an $R$-module $N$  any map $f: \prod_{i\in I}M_i = \{ \star\} \to N$ is multilinear. Put 
  $\bigotimes_{i\in I}M_i = R$ and let $\tau : \{ \star\} \to R$ be the map $\star \mapsto 1$. 
  Now let $\overline{f}: R \to N$ be the unique linear map such that $\overline{f}(1)= f(\star)$.
  Then $f = \overline{f}\circ \tau$ 
  and the pair $(R,\tau)$ fulfills the universal property of the tensor product. 
  
  If $I$ is a singleton with unique element $i_0$, then $\prod_{i\in I} M_i = M_{i_0}$ 
  and a map $f: \prod_{i\in I} M_i \to N$ is multilinear if and only if $f$ as a map from $M_{i_\circ}$ to $N$ 
  is linear. This implies that the pair $(M_{i_0},\id_{M_{i_\circ}})$ then is a tensor product 
  for the family $(M_i)_{i\in I}$. 
\item This is an immediate consequence of the universal property of the tensor product.
\item
  We construct an inverse to
  $\overline{\iota}_{J,1_J}: \bigotimes_{i\in I\setminus J}M_i \to  \bigotimes_{i\in I}M_i $.
  Let $x = (x_i)_{i\in I}$ be an element of $\prod_{i\in I}M_i$ and put
  \[
    \lambda (x) =  \left( \prod_{j\in J} x_j \right)\cdot \otimes_{i\in I\setminus J} x_i
    \left( \prod_{j\in J} x_j \right) \cdot \tau_{I\setminus J} ((x_i)_{i\in I\setminus J}) \ .
  \]
  Then $\lambda : \prod_{i\in I}M_i \to \bigotimes_{i\in \setminus J} M_i$
  is multilinear by construction, hence factors through a linear map
  $\overline{\lambda} : \bigotimes_{i\in I}M_i \to \bigotimes_{i\in I \setminus J} M_i$.
  By definition, $\overline{\lambda}$ is a left inverse of $\overline{\iota}_{J,1_J}$. It is also a
  right inverse since for all $(x_i)_{i\in I} \in \prod_{i\in I}M_i$  by multilinearity of $\tau_I$  
  \begin{equation*}
  \begin{split}
     \overline{\iota}_{J,1_J} \circ  \overline{\lambda}\circ  \tau_I  \left( (x_i)_{i\in I} \right) &  =
    \overline{\iota}_{J,1_J}
    \left( \left( \prod_{j\in J} x_j \right) \cdot  \otimes_{i\in I\setminus J} x_i \right)
    =\left( \prod_{j\in J} x_j\right)\cdot\left( \overline{\iota}_{J,1_J} \circ \tau_{I\setminus J}
    \left((x_i)_{i\in I\setminus J}\right)\right) = \\ & \hspace{-5em}
    =\left( \prod_{j\in J} x_j\right)\cdot\left( \tau_I \circ 
    \iota_{J,1_J}\left( (x_i)_{i\in I\setminus J}\right)\right) =
    \tau_I \circ \iota_{J,(x_j)_{j\in J}}\left( (x_i)_{i\in I\setminus J}\right) =
    \tau_I \left( (x_i)_{i\in I}\right) % = \otimes_{i\in I} x_i 
  \end{split}
  \end{equation*}
  and since by construction of the tensor product the image of $\tau_I$ is a generating system for
  the $R$-module $\bigotimes_{i\in I}M_i$.
\end{adromanlist}
\end{proof}

\begin{lemma}\label{thm:image-generating-system-canoncial-map-finite-tensor-product-generating-system} 
  Assume that $(M_i)_{i\in I}$ is a finite family of $R$-modules such that for every $i\in I$ a
  generating set $S_i$ of the $R$-module $M_i$ has been given. Then the set
  $S = \tau \left( \prod_{i\in I}S_i \right)$ is a generating set of the tensor product
  $\bigotimes_{i\in I} M_i$. 
\end{lemma}

\begin{proof}
  By construction of the tensor product in the proof of
  \Cref{thm:construction-fundamental-properties-infinite-tensor-product} it is clear that
  a generating set of $\bigotimes_{i\in I} M_i$ is given by the set
  of elements of the form $\otimes_{i\in I}x_i$ where $(x_i)_{i\in I}\in \prod_{i\in I}M_i$.
  Each of the $x_i$ can now be represented in the form
  \[
    x_i = \sum_{k=1}^{n_i} r_{i,k} s_{i,k} \quad\text{with}\enspace r_{i,1},\ldots ,r_{i,n_i}\in R,\enspace
    s_{i,1},\ldots ,s_{i,n_i}\in S_i \ .
  \]
  Hence, by multilinearity of $\tau$ and with $I=\{ i_1,\ldots ,i_d\}$,
  \[
    \otimes_{i\in I}x_i = \tau \left( (x_i)_{i\in I} \right) =
    \sum_{k_{i_1} =1}^{n_{i_1}}\cdots  \sum_{k_{i_d} =1}^{n_{i_d}} r_{i_1,k_{i_1}} \cdot \ldots \cdot r_{i_d,k_{i_d}}
    \cdot \tau \left( (s_{i,k_i})_{i\in I} \right)  \ ,
  \] 
  so $\otimes_{i\in I}x_i$ is a linear combination of elements of $S$ and the claim is proved. 
\end{proof}

\begin{lemma}\label{thm:componentwise-multilinear-maps-factorization}
  Let $(M_i)_{i\in I}$ be a family of $R$-modules, $(I_a)_{a\in A}$ a finite partition
  of the index set $I$, and $N$ an $R$-module. For $a\in A$ put
  $N_a = \bigotimes_{i\in I_a} M_i$ and let $\tau_a:  \prod_{i\in I_a} M_i \to  N_a$
  denote the canonical map. Assume that
  $f : \prod_{a\in A}\prod_{i\in I_a} M_i \to N$ is a map which is
  \emph{componentwise multilinear} in the following sense. 
  \begin{itemize}
  \item[$(\mathsf{ CM})$\hspace{-1mm}]
    Let $b\in A$ and $y=(y_a)_{a\in A}   \in \prod_{a\in A}\prod_{i\in I_a} M_i $
    a family with $y_b =0$. If for all $j\in I_b$ and families
    $x=(x_i)_{i\in I_b}\in \prod_{i\in I_b} M_i$ with $x_j  =0$ the  map
    \[
       M_j \to N , \enspace m \mapsto f( \iota_b (\iota_j (m) + x) + y)
    \]
    is linear, then $f$ factors through
    $(\tau_a)_{a\in A}: \prod_{a\in A}\prod_{i\in I_a} M_i \to
    \prod_{a\in A}N_a $. More precisely, there exists
    a unique multilinear map $\overline{f} : \prod_{a\in A}N_a \to N$ such that
    \[
       f = \overline{f} \circ (\tau_a)_{a\in A} \ .
    \]
  \end{itemize}
\end{lemma}

\begin{proof}
  We prove the claim by induction on the cardinality of $A$. 
  If $A$ is a singleton, then $\prod_{a \in A} \prod_{i\in I_a} M_i$ canonically
  coincides with $\prod_{i\in I} M_i$ and $f: \prod_{i\in I_a} M_i \to N$
  is multilinear, hence by the universal property of the tensor product there
  exists a unique linear map $\overline{f}: N_a \to N$ such that
  $f = \overline{f} \circ \tau_a$. 

  Now assume that the claim holds whenever the cardinality of the index set $A$ is
  $\leq n$ for some $n \in \N^*$.
  Assume to be given initial data $(M_i)_{i\in I}$ and $§N$, a partition
  $(I_a)_{a\in A}$ of $A$ with $|A| = n+1$ and  componentwise multilinear map
  $f: \prod_{a\in A}\prod_{i\in I_a} M_i \to N$. Fix $a \in A$ and put
  $B =A \setminus \{ a\}$. Let
  $x = (x_i)_{i\in I_a} \in \prod_{i\in I_a}M_i$
  and $\widetilde{x}$ be the element of $ \prod_{d\in A}\prod_{i\in I_d} M_i $ such that
  \[
    \pi_d (\widetilde{x}) =
    \begin{cases}
      x &\text{for} \enspace d = a \ , \\
      0 &\text{else} \ .
    \end{cases}
  \]
  The map
  \[
    f_x: \prod_{b\in B}\prod_{i\in I_b} M_i \to N , \enspace
         y  \mapsto f (\iota_B (y) + \widetilde{x}) 
  \]
  then is componentwise multilinear. Hence by inductive assumption there  exists
  a unique multilinear map $\overline{f_x} : \prod_{b\in B}N_b \to N$ such that
  $f_x = \overline{f_x}\circ (\tau_b)_{b\in B}$. By assumption on $f$
  the map
  $\prod_{i\in I_a} M_i \to \Map \left( \prod_{b\in B} \prod_{i\in I_b} M_i,N\right) $,
  $x \mapsto f_x$ is multilinear which implies multilinearity of
  \[
    \overline{f_\bullet} :
    \prod_{i\in I_a} M_i \to \mlinOps \left( \prod_{b\in B} N_b ,N \right),
    \enspace x \mapsto \overline{f_x} \ .
  \]
  Let $F: N_a \to \mlinOps \left( \prod_{b\in B} N_b,N \right)$ be its
  linearization. Application of the exponential law for multilinear maps,
  \Cref{thm:exponential-law-multilinear-maps}, now gives a
  multilinear map $\eta (F) : \prod_{d \in A} N_d \to N$ which we denote 
  by $\overline{f}$. Given a family $(x_d)_{d\in A}$ of families
  $x_d =(x_i)_{i\in I_d}$ one checks
  \[
    \overline{f} \left( \big(\tau_d (x_d)\big)_{d\in A} \right)
    = F \big( \tau_a(x_a) \big) \left( \big(\tau_b (x_b)\big)_{b\in B} \right)
    = \overline{f}_{x_a} \left( \big(\tau_b (x_b)\big)_{b\in B} \right)
    = f_{x_a} \left( (x_b)_{b\in B} \right) = f  \left( (x_d)_{d\in A} \right) \ .
  \]
  Hence $\overline{f} \circ (\tau_d)_{d\in A} =f$.
  To finish the induction step it remains to prove uniqueness.
  So let $\overline{g} :  \prod_{d \in A} N_d \to N$ be another multilinear map
  such that $\overline{g} \circ (\tau_d)_{d\in A} =f$  and consider
  the induced linear map
  $\overline{g}^\sharp =\eta^{-1} (\overline{g}) : N_a \mapsto \mlinOps (\prod_{b\in B}N_b, N)$.
  Then for every $x\in \prod_{i\in I_a}M_i$ the relation 
  \[
    \overline{g}^\sharp (\tau_a(x)) \circ (\tau_b)_{b\in B} = f_x =
    \overline{f}_x \circ (\tau_b)_{b\in B} 
  \]
  is satisfied. 
  Hence $\overline{g}^\sharp (\tau(x)) = \overline{f}_x$ for all $x\in \prod_{i\in I_a}M_i$ which entails
  that $\overline{g}^\sharp$ coincides with $F$. By
  \Cref{thm:exponential-law-multilinear-maps} one obtains
  $\overline{g} = \overline{f}$. This finishes the induction step and the lemma
  is proved. 
\end{proof}

\begin{proposition}
  Let $(M_i)_{i\in I}$ be a family of $R$-modules and $(I_a)_{a\in A}$ a finite partition of the index set $I$.
  Then there exists a natural isomorphism
  \[
    \alpha_{I,A}: \bigotimes_{i \in I} M_i \to \bigotimes_{a \in A}  \bigotimes_{i \in I_a} M_i .
  \]  
\end{proposition}

\begin{proof}
  Put $N_a = \bigotimes_{i \in I_a} M_i$  for  $a \in A$ and let
  $\tau_a :  \prod_{i \in I_a} M_i \to  N_a$ be the canonical map to the tensor product.
  Let $\tau_A :\prod_{a \in A}N_a \to \bigotimes_{a \in A} N_a$ be the canonical map to
  the tensor product of the modules $N_a$.
  Define $\tau_{I,A}: \prod_{i \in I} M_i\to \prod_{a \in A} N_a$ as the unique map so that
  $\pi_a  \circ \tau_{I,A} = \tau_a \circ \pi_{I_a}$  for all $a\in A$.
  % As before, the map $\pi_a$ on the right hand side is hereby the unique map
  % $\pi_{I_a} : \prod_{i\in I} M_i\to\prod_{i\in I_a} M_i $
  % such that $\pi_k \circ \pi_{I_a} =\pi_k$ for all $k\in I_a$.
  By construction $\tau_{I,A} = (\tau_a)_{a\in A} \circ \kappa_{I,A}$,
  where $\kappa_{I,A} :  \prod_{i\in I} M_i \to \prod_{a \in A} \prod_{i \in I_a} M_i$ is the natural isomorphism from
  \Cref{thm:associator-cartesian-product}.  
  The composition $\tau_A \circ \tau_{I,A}$ then is multilinear by \Cref{thm:construction-multilinear-maps-composition}
  \ref{ite:multilinearity-composition-multilinear-map-product-multilinear-map}, hence factors through a linear map
  $\alpha_{I,A} : \bigotimes_{i\in I} M_i \to \bigotimes_{a \in A} N_a$ 
  that is
  \begin{equation}\label{eq:defining-equation-associator-map-tensor-product}
    \tau_A \circ (\tau_a)_{a\in A} \circ \kappa_{I,A} =
    \alpha_{I,A} \circ \tau_I \ .
  \end{equation}  
     
  Naturality of $\alpha_{I,A}$ in  $(M_i)_{i\in I}$ is clear by definition so it remains to
  construct an inverse to $\alpha_{I,A}$.  Consider the composition
  $\tau_I \circ \kappa^{-1}: \prod_{a \in A} \prod_{i \in I_a} M_i
   \to \bigotimes_{i\in I} M_i$. Assume that $a \in A$ and
  $(y_b)_{b\in A\setminus\{a\}} \in \prod_{b \in A\setminus\{a\}} \prod_{i \in I_b} M_i$
  have been chosen. Let $y_a\in \prod_{i \in I_a} M_i$ be $0$, put
  $\widetilde{y} =(y_d)_{d\in A} \in \prod_{d \in A}  \prod_{i \in I_d} M_i$, and let
  $y\in  \prod_{i \in I} M_i$ be the family such that $\pi_i(y) = \pi_i (y_{a(i)})$ for
  all $i\in I$, where $a(i)$ denotes the unique element of $A$  such that
  $i\in I_{a(i)}$. In other words let $y =\kappa^{-1} (\widetilde{y})$.
  For every $j\in I_a$ and $x= (x_i)_{i\in I_a}\in \prod_{i\in I_a}M_i$
  with $\pi_j (x)=0$ the map
  \[
    M_j \to \bigotimes_{i\in I} M_i,\enspace m \mapsto \tau_I \circ \kappa^{-1} \left( \iota_a (\iota_j(m)+x)+\widetilde{y}\right)
    = \tau_I \left( \iota_j (m) + \iota_{I_a}(x) + y \right)
  \]
  then is multilinear since $\tau_I$ is multilinear and $\pi_j(\iota_{I_a}(x) + y)=\pi_j(x) +\pi_j(y_a) = 0$.
  Hence $\tau_I \circ \kappa^{-1}$ is componentwise multilinear and therefore,
  by \Cref{thm:componentwise-multilinear-maps-factorization}, factors
  through the map
  $(\tau_a)_{a\in A} : \prod_{a\in A} \prod_{i\in I_a} M_i\to \prod_{a\in A} N_a$
  which means that
  \begin{equation}\label{eq:defining-equation-inverse-associator-map-tensor-product}
    \tau_I \circ \kappa^{-1} = \lambda_{I,A}\circ (\tau_a)_{a\in A}
  \end{equation}
  for some uniquely defined multilinear map
  $\lambda_{I,A} : \prod_{a\in A} N_a\to\bigotimes_{i\in I} M_i$.
  Let
  \[ \overline{\lambda}_{I,A} : \bigotimes_{a\in A} N_a\to\bigotimes_{i\in I} M_i\]
  be the linearization of $\lambda_{I,A}$.
  We claim that $\overline{\lambda_{I,A}}$ is inverse to $\alpha_{I,A}$.
  By definition of $\overline{\lambda_{I,A}}$ and 
  Eqs.~\eqref{eq:defining-equation-associator-map-tensor-product} and
  \eqref{eq:defining-equation-inverse-associator-map-tensor-product} one concludes
  %
  \[
    \overline{\lambda_{I,A}} \circ \alpha_{I,A} \circ \tau_I =
    \overline{\lambda_{I,A}} \circ \tau_A \circ (\tau_a)_{a\in A} \circ \kappa_{I,A} =
    \lambda_{I,A} \circ (\tau_a)_{a\in A} \circ \kappa_{I,A} = \tau_I \ . 
  \]
  Since the image of $\tau_I$ generates $\bigotimes_{i\in I} M_i$ as an $R$-module,
  $\overline{\lambda_{I,A}}$ has to be left inverse to $\alpha_{I,A}$.
  Using Eqs.~\eqref{eq:defining-equation-associator-map-tensor-product} and
  \eqref{eq:defining-equation-inverse-associator-map-tensor-product} again compute
  \[
    \alpha_{I,A} \circ\overline{\lambda_{I,A}} \circ \tau_A \circ (\tau_a)_{a\in A} =
    \alpha_{I,A} \circ \lambda_{I,A} \circ (\tau_a)_{a\in A} = \alpha_{I,A}\circ\tau_A\circ \kappa_{I,A}^{-1}
    =  \tau_A \circ (\tau_a)_{a\in A} \ .    
  \]
  Since by \Cref{thm:image-generating-system-canoncial-map-finite-tensor-product-generating-system}
  the image of
  $\tau_A \circ (\tau_a)_{a\in A}$ generates $\bigotimes_{a \in A}  \bigotimes_{i \in I_a} M_i$,  the equality
  \[ \alpha_{I,A} \circ\overline{\lambda_{I,A}}=\id_{\bigotimes_{a \in A}  \bigotimes_{i \in I_a} M_i} \]
  follows and the proposition is proved.
\end{proof}

\begin{propanddef}
  Let $(A_i)_{i\in I}$ be a family of $R$-algebras. Then the tensor product
  $A = \bigotimes_{i\in I} A_i$ carries in a natural way the structure of an
  $R$-algebra where the product map is defined by
  \[
    \cdot :  A \times A \to A , \enspace
    (\otimes_{i\in I} a_i , \otimes_{i\in I} b_i)\mapsto
    \otimes_{i\in I} (a_i\cdot b_i) \ .
  \]
  In case each of the algebras $A_i$ is commutative, then $A$ is commutative
  as well.
  Likewise, if each $A_i$ is unital and $1_i$ denotes the unit element of
  $A_i$, then $A$ is unital with unit given by $1 = \otimes_{i\in I} 1_i$.
  One calls $A$ the \emph{tensor product algebra} of the family of algebras
  $(A_i)_{i\in I}$.
\end{propanddef}

\begin{proof}
  The map
  \[
    \prod_{(i,k)\in I \times \{ 1,2\}} A_i \to A,\enspace
    (a_{i,k})_{(i,k) \in I\times \{1,2\}} \mapsto\otimes_{i\in I}(a_{i,1}\cdot a_{i,2})
  \]
  is multilinear by bilinearity of the product maps on the $A_i$ and multilinearity of $\tau_I$,
  so factors through a linear map
  $\mu: A \otimes A \cong \bigotimes_{(i,k)\in I\times \{1,2\}} A_i \to A$. Composition of
  $\mu$ with the canonical bilinear map $A \times A \to A\otimes A$ gives the product map
  $\cdot : A \times A\to A$
  and shows that the product on $A$ is well-defined. By construction, the product map $\cdot$
  is bilinear. Given $\otimes_{i\in I} a_i, \otimes_{i\in I} b_i, \otimes_{i\in I} c_i \in A$ one
  computes
  \[
    \big( \otimes_{i\in I} a_i \cdot \otimes_{i\in I} b_i \big) \cdot \otimes_{i\in I} c_i
    = \otimes_{i\in I} ((a_i\cdot b_i)\cdot c_i) =
    \otimes_{i\in I} (a_i\cdot( b_i\cdot c_i)) =
    \otimes_{i\in I} a_i \cdot \big( \otimes_{i\in I} b_i  \cdot \otimes_{i\in I} c_i\big) \ .
  \]
  This entails that the product on $A$ is associative. In the same way one shows
  that $A$ is commutive respectively unital if each of the $A_i$ is. 
\end{proof}

\para
As we have seen, the infinite tensor product construction works well for objects of
algebraic categories like $R$-modules, vector spaces or $R$-algebras. As soon as
a topologies compatible with the algebraic structure come in it becomes difficult and
sometimes even impossible to construct or even define 
